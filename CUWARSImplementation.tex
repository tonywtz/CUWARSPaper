\documentclass[12pt]{article}

\marginparwidth 0.5in
\oddsidemargin 0.25in
\evensidemargin 0.25in
\marginparsep 0.25in
\topmargin 0.25in
\textwidth 6in
\textheight 8 in

\usepackage{setspace,graphicx,epstopdf,amsmath,amsfonts,amssymb,amsthm}
\usepackage{marginnote,datetime,enumitem,subfigure,rotating}
\usepackage{float} \usepackage[longnamesfirst]{natbib}

\newtheorem{condition}{Condition} \newtheorem{remark}{Remark}
\newtheorem{corollary}{COROLLARY}
\newtheorem{proposition}{PROPOSITION} \newtheorem{obs}{OBSERVATION}
\newcommand{\argmax}{\mathop{\rm arg\,max}}
\newcommand{\sign}{\mathop{\rm sign}}
\newcommand{\defeq}{\stackrel{\rm def}{=}}


\setlength{\hoffset}{-0.4in}
\setlength{\textwidth}{6.8in}
\setlength{\textheight}{9.3in}
\setlength{\topmargin}{-0.5in}


\begin{document}

\author{}
\title{A Risk-based Theory of Exchange Rate Stabilization: \\
  Implementation with Intermediaries}

\maketitle

\setstretch{1.5}

\noindent
There are two discrete time periods, $t = 1, 2$. There exists a unit
measure of households $i \in [0, 1]$, partitioned into three subsets
$\Theta^n$ of measure $\theta^n$. Each subset represents the
constituent agents of a country. We label these countries
$n = \left\{ m, t, o \right\}$ for the stabilizing (manipulating),
target and outside countries, respectively. Agents make an investment
decision in the first period. All consumption occurs in the second
period.

Within a country $n$, a fraction $1 - \phi^n$ of agents are households
and do not have direct access to international financial markets.
Households can only save in the form of real bonds. The remaining
fraction $\phi^n$ of agents run financial intermediaries. Financial
intermediaries can invest in both domestic and foreign real bonds. In
the first period, households save by transferring all of their wealth
to the financial intermediary in return for a real domestic bond.

All agents derive utility from a consumption index comprising a
country-specific non-traded good, $C_N$, and a traded good, $C_T$:
\begin{equation*}
  C(i) = C_T(i)^{\tau} C_N(i)^{1 - \tau}
\end{equation*}
and $\tau \in (0, 1)$. Agents exhibit constant relative risk aversion
according to:
\begin{equation}
  U(i) = \frac{1}{1 - \gamma} 
  \mathbb{E}\left[  \left( C(i) \right)^{1 - \gamma} \right]
  \label{eqn:utility}
\end{equation}
where $\gamma > 0$ is the coefficient of relative risk aversion.

\subsection*{Firms}

In each country $n$, there also exists a mass $\theta^n$ of firms that
produce their own country-specific non-traded good using a
Cobb-Douglas production technology that employs capital and labor.
Each firm is endowed with a unit of capital in the first period.
Capital goods can only be freely shipped in the first period when they
are invested for use in the production of non-traded goods in the
second period. The capital good fully depreciates after production
occurs in the second period. Each household supplies one unit of labor
inelastically towards production in the second period. The per capita
output of non-traded goods is
\begin{equation*}
  Y_N^n  = \exp \left( \eta^n \right) \left( K^n \right)^{\nu}
\end{equation*}
where $0 < \nu < 1$ is the capital share in production, $K^n$ is the
{\emph per capita} stock of capital in country $n$ and $\eta^n$ is a
country-specific productivity shock realized at the start of the
second period,
\begin{equation*}
  \eta^n \sim N\left( - \frac{1}{2} \sigma_N^2, \sigma_N^2 \right).
\end{equation*}

\subsection*{Cash-in-Advance}

To formally introduce currencies, all agents face a cash-in-advance
constraint -- They must use their domestic currency when buying assets
in period 1, and when buying consumption goods in period 2. Each
central bank controls its own money supply. Let $\Delta M^n_1$ and
$\Delta M^n(\omega)$ denote the growth in the money supply in the
first and second periods, respectively.

In the second period, (inactive) households receive their bond payoff
from the intermediaries and consume:
\begin{equation*}
  \tilde{P}^n_T \hat{C}^n_T(\omega) + \tilde{P}^n_N \hat{C}^n_N(\omega)
  \le \tilde{P}^n.
\end{equation*}
The intermediary faces the following cash-in-advance constraint:
\begin{equation*}
  \tilde{P}^n_T C^n_T(\omega) + P^n_N \hat{C}^n_N(\omega)
  + \frac{1 - \phi^n}{\phi^n} \tilde{P}^n(\omega)
  \le \tilde{M}^n_2(\omega).
\end{equation*}
where $\tilde{M}^n_2(\omega)$ is the total quantity of currency
available to the country $n$ intermediary to use to spend on
consumption and to transfer to the household. This quantity of
currency comprises the monetary payoff from the intermediary's bond
portfolio and any additional money balances the central bank injects
through open market operations:
\begin{equation*}
  \Delta \tilde{M}^n(\omega)
  = \tilde{M}^n_2(\omega)
  - \left(
    \frac{1}{\phi^n} \left(
      \tilde{P}^n_T(\omega) Y^n_T + \tilde{P}^n_N(\omega) Y^n_N(\omega)
    \right) + \sum_\ell B^n_\ell \tilde{P}^n(\omega) 
  \right).
\end{equation*}

The central bank in each country controls the money supply through
$\Delta M^n(\omega)$ in order to stabilize the nominal exchange rate.

\subsection*{Equilibrium}

The intermediary's first order conditions with respect to traded and
non-traded goods are standard. Because the utility functions are
Cobb-Douglas, intermediaries and households spend a fraction $\tau$ of
their total wealth on traded goods and a fraction $1 - \tau$ on
non-traded goods. Intermediaries consume:
\begin{align}
  C^n_T(\omega)
  & = \tau \frac{
    \tilde{M}^n_2(\omega) - ((1 - \phi^n) / \phi^n) \tilde{P}^n(\omega)
    }{\tilde{P}^n_T(\omega)}  \\
  C^n_N(\omega)
  & = (1 - \tau) \frac{
    \tilde{M}^n_2(\omega) - ((1 - \phi^n) / \phi^n) \tilde{P}^n(\omega)
    }{\tilde{P}^n_N(\omega)}
\end{align}
Household consumption is given by:
\begin{align}
  \hat{C}^n_T(\omega)
  & = \tau \frac{\tilde{P}^n(\omega)}{\tilde{P}^n_T(\omega)}  \\
  \hat{C}^n_N(\omega)
  & = (1 - \tau) \frac{\tilde{P}^n(\omega)}{\tilde{P}^n_N(\omega)}
\end{align}
The market clearing conditions for traded and non-traded goods are:
\begin{equation}
  \sum_n \theta^n \left( \phi^n C^n_N(\omega) + (1 - \phi^n) \hat{C}^n_N(\omega)  \right) = 1,
\end{equation}
and
\begin{equation}
  \phi^n C^n_N(\omega) + (1 - \phi^n) \hat{C}^n_N(\omega) = Y^n_N.
\end{equation}



\end{document}


