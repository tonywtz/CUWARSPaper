\section{Additional Empirical Results \label{Appendix_Empirics}}

\begin{table}[htp!]
  \caption{Interest Rate Differentials, Capital Intensity, and
    Exchange Rate Regimes \vspace{-1em}}
  \label{table:stylized}
  \begin{center}
    \begin{tabular}{l c c c }
      \hline
      \hline
      Panel A: & \multicolumn{3}{c}{\emph{Table Corresponding to Figure \ref{fig_stylized}}} \\
      \cline{2 - 4} \vspace{-1em} \\
               & Interest Rate Diff. & FX Excess Returns & Capital Intensity \\
      \hline
      FX Allowed Deviation            & $0.41^{***}$   & $0.35^{**}$   & $-1.56^{**}$     \\
               & $(0.13)$       & $(0.15)$      & $(0.74)$       \\
      GDP Share & $-23.36^{***}$ & $-12.50^{**}$ & $86.82^{***}$ \\
               & $(7.54)$       & $(5.85)$      & $(27.83)$       \\
      \hline
      Num. obs.               & 7,515           & 7,515          & 662            \\
      R$^2$       & 0.12           & 0.00          & 0.06           \\
      \hline
      Panel B: & \multicolumn{3}{c}{\emph{Table with Alternative Measure of Stabilization}} \\
      \cline{2 - 4} \vspace{-1em} \\
               & Interest Rate Diff. & FX Excess Returns & Capital Intensity \\
      \hline
      $\zeta$ & $-4.08^{***}$  & $-3.51^{**}$  & $15.58^{**}$    \\
               & $(1.28)$       & $(1.49)$      & $(7.41)$       \\
      GDP Share & $-23.36^{***}$ & $-12.50^{**}$ & $86.82^{***}$ \\
               & $(7.54)$       & $(5.85)$      & $(27.83)$       \\
      \hline
      Num. obs.               & 7,515           & 7,515          & 662            \\
      R$^2$       & 0.12           & 0.00          & 0.06           \\
      \hline
      Panel C: & \multicolumn{3}{c}{\emph{Summary Statistics}} \\
      \cline{2 - 4} \vspace{-1em} \\ 
               & & Mean & Std. Dev \\
      \hline
      Interest Rate Diff. (\%) &  & 1.95 & 4.56 \\ 
      FX Excess Returns (\%) &  & 2.29 & 36.44  \\ 
      Capital Intensity (\%) & & 118.86 & 24.12 \\ 
      FX Allowed Deviation (\%) & & 3.54 & 3.65 \\ 
      $\zeta$ & & 0.65 & 0.37 \\ 
      GDP Share & & 0.04 & 0.07  \\ 
      \hline
      \hline
    \end{tabular}
  \end{center}
  \small\textit{Notes: } This table shows panel regressions of
  interest rate differentials, currency excess returns, and capital
  intensity on the allowed annual standard deviation under the issuing
  country's exchange rate regime and country size (the share the
  issuing country contributes to World GDP). The sample and data
  construction correspond to Figure \ref{fig_stylized}. Panel A shows
  the full regression output corresponding to the conditional scatter
  plots in Figure \ref{fig_stylized}. Panel B shows an alternative
  specification that follows more closely the model by using
  $\zeta=1-(FX\text{}Allowed\text{ }Deviation)/10 $ as a regressor,
  where we again assume that a freely floating exchange rate amounts
  to an allowed annual standard deviation of 10\%. Standard errors are
  clustered by currency. Panel C displays summary statistics. Capital
  intensity is the log capital to output ratio. Interest Rate Diff.,
  FX Excess Returns, Capital Intensity, and FX Allowed Deviation are
  all given in percent. See the caption of Figure \ref{fig_stylized}
  for details.
\end{table}


\clearpage

\section{Appendix to Section \ref{sec_ReducedFormResults}
  \label{Appendix_ReducedFormResults}}

The country $n$ risk-free bond pays off $P^n$ units of the traded good
at maturity. We derive the value of the risk-free bond, $V^n_P$, by
applying the asset pricing equation to the bond payoff:
\begin{equation*}
  V^n_P = \mathbb{E}\left[\Lambda_{T} P^n
  \right],
\end{equation*}
where $\Lambda_{T}$ denotes the stochastic discount factor. The
country $n$ risk-free rate (in levels), $R^n$, is the inverse of the
price of the risk-free bond:
\begin{equation*}
  R^n = \frac{1}{V^n_P}.
\end{equation*}
Putting the previous two equations together yields the following
relationship:
\begin{equation*}
  \mathbb{E}\left[ \Lambda_{T} P^n \right] R^n = 1.
\end{equation*}
As a result, the risk-free rates of countries $f$ and $h$ are related
as follows:
\begin{equation*}
  \mathbb{E}\left[\Lambda_{T} P^f \right] R^f
  = \mathbb{E}\left[\Lambda_{T} P^h \right] R^h = 1
\end{equation*} 
If the stochastic discount factor and prices are log-normal, we can
perform the following calculations:
\begin{align*}
  & \mathbb{E}\left[\Lambda_{T} P^f \right] R^f
    = \mathbb{E}\left[\Lambda_{T} P^h \right] R^h \\
  \Leftrightarrow\quad
  & \mathbb{E}\left[\exp\left[ \lambda_{T} + p^f + r^f \right]\right]
    = \mathbb{E}\left[\exp\left[\lambda_{T} + p^h + r^h \right]\right] \\
  \Leftrightarrow\quad
  & \mathbb{E}\left[\lambda_{T} + p^f\right] + \frac{1}{2}\text{var}\left(\lambda_{T}\right) + 
    \frac{1}{2}\text{var}\left(p^f\right) + \text{cov}\left(\lambda_{T}, p^f\right) + r^f \\
  & = 
    \mathbb{E}\left[\lambda_{T}+ p^h\right] + \frac{1}{2}\text{var}\left(\lambda_{T}\right)+ 
    \frac{1}{2}\text{var}\left(p^h\right) + \text{cov}\left(\lambda_{T}, p^h\right)+r^h,
\end{align*}
We cancel out $\text{var}\left( \lambda_{T} \right)$ from both sides
of the previous equation.
\begin{align*}
  &\mathbb{E}\left[p^f\right]+\frac{1}{2}\text{var}\left(p^f\right) + \text{cov}\left(\lambda_T, p^f\right)+r^f
    = \mathbb{E}\left[p^h\right]+\frac{1}{2}\text{var}\left(p^h\right)+\text{cov}\left(\lambda_T,p^h\right)+r^h
  \\      \Leftrightarrow \quad
  & r^f+\mathbb{E}\left[p^f-p^h\right]+\frac{1}{2}\text{var}\left(p^f\right)-\frac{1}{2}\text{var}\left(p^h\right)-r^h = -\text{cov}\left(\lambda_T,p^f-p^h\right)\\   
  \Leftrightarrow\quad
  &r^f+\log\left(\mathbb{E}\left[P^f\right]/\mathbb{E}\left[P^h\right]\right)-r^h = -\text{cov}\left(\lambda_T,p^f-p^h\right)
\end{align*}
We define
$\Delta\mathbb{E}\left[s^{f,h}\right]
=\log\left(\mathbb{E}\left[P^f\right]/\mathbb{E}\left[P^h\right]\right).$
With this definition:
\begin{equation*}
  r^f+\Delta\mathbb{E}\left[s^{f,h}\right]-r^h = -\text{cov}\left(\lambda_T,p^f-p^h\right).
\end{equation*}


\section{Appendix to Sections \ref{sec:environment} -
\ref{sec:positive_effects}}


\subsection{Equilibrium Conditions
  \label{Appendix_ModelDetails}}

In this appendix, we provide additional details for our baseline model
in Section \ref{sec:environment} and formally derive its equilibrium
conditions. To avoid solving the optimization problem separately for
households in the stabilizing country and households in the rest of
the world, we generalize the notation to allow all countries to impose
state-contingent taxes, $Z^n(\omega)$, and provide lump sum transfers,
$\bar{Z}^n$. The governments in the target and outside countries do
not use these instruments, such that $Z^t(\omega)=Z^o(\omega) = 1$ and
$\bar{Z}^t =\bar{Z}^o= 0$.

In the second period, all households maximize their utility
(\ref{eqn:utility}) subject to their budget constraint:
\begin{equation}
  Z^n(\omega) C_T^n(\omega) + P_N^n(\omega) C_N^n(\omega)  
  \le \sum_l A_l^n P_N^l(\omega) Y_N^l(\omega) + Y_T 
  \label{eqn:budget_cons_2}
\end{equation}
where $P^n_N(\omega)$ is the price of the nontraded good in the
stabilizing country in state $\omega$, $A^n_l$ is number the stocks a
country $n$ household owns in the firm in country $l$, and $Y_T = 1$
is the unit endowment of the traded good.

In the first period, households choose their portfolio of stocks to
maximize expected utility in the second period. The first-period
budget constraint reads:
\begin{equation}
  \sum_l A_l^n Q^l_N + Q_K K^n_N \le W^n_0.
  \label{eqn:budget_cons_1}
\end{equation}
where $W^n_0$ represents initial household wealth in terms of traded
goods in the first period.

$\Lambda_T^n(\omega)$ denotes the Lagrange multiplier on the budget
constraint for the country $n$ household in state $\omega$ in the
second period. The first-order conditions are:
\begin{align}
  \frac{\tau \left( C^n(\omega)^{1 - \gamma} \right) 
  \left( C_T^n(\omega) \right)^{-1}}{Z^n(\omega)}
  & = \Lambda^n_T(\omega) \label{eqn:FOCCT_full}\\ 
  (1 - \tau) \left( C^n(\omega)^{1 - \gamma} \right) 
  \left( C_N^n(\omega) \right)^{-1} & = \Lambda^n_T(\omega) P^n_N(\omega).
                                      \label{eqn:FOCCN_full}
\end{align}
The consumption tax drives a wedge between the marginal utility of
consumption of traded goods and its shadow price, as equation
\eqref{eqn:FOCCT_full} shows. Equations \eqref{eqn:FOCCN} and
\eqref{eqn:FOCCN_full} jointly imply
$P_N^n(\omega)= \Lambda_N^n(\omega) / \Lambda^n_T(\omega)$.

Next, we derive equilibrium conditions that determine first-period
investment in stocks and capital. Since the final consumption bundle
is a Cobb-Douglas aggregate of traded and nontraded goods, households
spend a fraction $\tau$ of their second-period wealth on traded
consumption and a fraction $1 - \tau$ on nontraded consumption:
\begin{align*}
  C^n_T(\omega) = \tau
  \left( \frac{\sum_l A^n_l P^l_N(\omega) Y^l_N(\omega) + 
  Y_T}{Z^n(\omega)} \right) \text{ and }
  C^n_N(\omega) = (1 - \tau)
  \left( \frac{\sum_l A^n_l P^l_N(\omega) Y^l_N(\omega) + 
  Y_T}{P^n_N(\omega)} \right).
\end{align*}
In the first period, households choose their portfolio of stocks and
firms decide on their capital investment, $K^n_N$. We plug the
consumption of traded and nontraded goods into equations
(\ref{eqn:cesutil}) and (\ref{eqn:utility}) and take first-order
conditions to obtain:
\begin{equation*}
  Q^l_N
  = \mathbb{E}\left[
    \left( \frac{\tau^\tau (1 - \tau)^{1 -\tau}}{\Lambda_{T, 1}^n \left( Z^n(\omega) \right)^{\tau}
        \left(P^n_N(\omega) \right)^{1 - \tau}} \right)
    \left( C^n(\omega) \right)^{- \gamma} P^l_N(\omega) Y^l_N(\omega) \right],
\end{equation*}
where $\Lambda_{T, 1}^n$ denotes the Lagrange multiplier on the
first-period budget constraint for a household in country $n$.


Divide through by $Q^l_N$ and apply the definition of the price index
$P^n(\omega)$ given by equation \eqref{eqn:price_index} in Appendix
\ref{Appendix_PriceIndex} to obtain
\begin{equation}
  \mathbb{E}\left[ \frac{\Lambda_T^n(\omega)}{\Lambda_{T, 1}^n} 
    \frac{P^l_N(\omega) Y^l_N(\omega)}{Q_N^l} \right] = 1.
  \label{eqn:FOC_Stock}
\end{equation}

Firms invest in capital to maximize the expected discounted value of
profits:
\begin{equation*}
  \max_{K^n} \mathbb{E}\left[ 
    \left( \frac{\Lambda^n_T(\omega)}{\Lambda^n_{T, 1}} \right) 
    P_N^n(\omega) \exp\left( \eta^n \right) \left( K^n \right)^{\nu} \right] - 
  Q_K \left( K^n - 1 \right).
\end{equation*}
Their first-order condition with respect to $K^n$
yields\begin{equation*} Q_K = \nu \mathbb{E}\left[
    \frac{\Lambda_T^n(\omega)}{\Lambda_{T, 1}^n} P^n_N(\omega)
    \exp\left( \eta^n \right) \left(K^n\right)^{\nu - 1} \right].
\end{equation*}
Multiply both sides of the previous equation by $K^n$, divide by
$Q_K$, substitute $Y_N^n = \exp(\eta^n) \left( K^n \right)^\nu$, and
apply the definition of $P^n_N(\omega)$ to get (\ref{eqn:FOCK}).

Equations \eqref{eqn:FOC_Stock} and \eqref{eqn:FOCK} show
$\Lambda^n_T(\omega) / \Lambda^n_{T, 1}$ are the stochastic discount
factors used to price assets that pay off in traded goods in the
second period. Since stocks and capital are freely traded in
international markets, all households must be marginal to investing in
all stocks and all firms must be marginal to purchasing an additional
unit of capital. As a result, the stochastic discount factors are
equal in equilibrium across countries,
\begin{equation}
  \frac{\Lambda^n_T(\omega)}{\Lambda^n_{T, 1}} = 
  \frac{\Lambda^m_T(\omega)}{\Lambda^m_{T, 1}} \quad
  \forall ~n, m,
  \label{eqn:sdf_equal}
\end{equation}
even though the government's intervention drives a wedge between
$\Lambda_T(\omega)$ and the marginal utility of traded consumption in
the stabilizing country, as equation \eqref{eqn:FOCCT_full} shows.

As a final step, we derive the equations that pin down the first and
second-period Lagrange multipliers. Household wealth in the first
period is:
\begin{equation*}
  W^n_0 = Q^n_N + Q_K + \kappa^n + \bar{Z}^n,
\end{equation*}
Recall that households are endowed with a unit of stock and a unit of
capital. $\kappa^n$ is the transfer that equalizes the marginal
utility of wealth across households when countries do not manipulate
the exchange rate, and the transfer $\bar{Z}^m$ ensures the same is
true under a stabilization, so that\begin{equation} \Lambda^{n}_{T, 1}
  = \Lambda^{}_{T, 1} \quad \forall ~n.
  \label{eqn:mult_equal}
\end{equation}
As a result, \eqref{eqn:sdf_equal} implies
\begin{equation*}
  \Lambda^n_T(\omega) = \Lambda_T(\omega)
  \quad \forall ~n, \omega.
\end{equation*}

Hence, we drop the country index on the Lagrange multipliers, and
interpret $\Lambda_T(\omega)$ as the shadow price of traded
consumption in the target and outside countries in the second period.
This result implies equations \eqref{eqn:FOCCT}, \eqref{eqn:FOCCN} and
\eqref{eqn:FOCK}.

Equation \eqref{eqn:mult_equal} shows the first-period Lagrange
multipliers are equal to each other, but it does not determine the
level of the Lagrange multipliers. Without loss of generality, we
normalize the first period Lagrange multiplier:
\begin{equation}
  \Lambda_{T, 1} =  \mathbb{E}\left[ \Lambda^n_T(\omega) \right].
  \label{eqn:norm_lambda}
\end{equation}


\subsection{Deriving the Price Index \label{Appendix_PriceIndex}}
The cost of one unit of consumption in country $n$ is given by the price index
\begin{equation}
  P^n = \arg \min C_T^n + P_N^n C_N^n \text{ s.t. } 
  \left( C_T^n \right)^\tau \left( C_N^n \right)^{1 - \tau} = 1.
  \label{eqn:InactiveProblem}
\end{equation}
First-order conditions imply
$C^n_N = \left( 1 - \tau \right) / \left( P^n_N \tau \right) C^n_T$. We plug
this expression for $C_N^n$ into the constraint
$\left( C_T^n \right)^\tau \left( C_N^n \right)^{1 - \tau} = 1$, and
solve for $C_T^n$:
\begin{equation*}
  C_T^n = \left( \frac{\tau}{1 - \tau} P_N^n \right)^{1 - \tau}.
\end{equation*}
We plug the expressions for $C_T^n$ and $C_N^n$ back into equation
(\ref{eqn:InactiveProblem}) to derive the optimal price index:
\begin{equation}
  P^n = \frac{ \left( P_N^n \right)^{1 - \tau}}{\tau^\tau (1 - \tau)^{1 - \tau}}.
  \label{eqn:price_index}
\end{equation}
The total value of consumption for households in country $n$ is
\begin{equation*}
  P^n C^n = 
  \left( \frac{ \left( P_N^n \right)^{1 - \tau}}{\tau^\tau (1 - \tau)^{1 - \tau}} \right)
  \left( \left( C_T^n \right)^\tau \left( C_N^n \right)^{1 - \tau} \right) = 
  \frac{C_T^n}{\tau}.
\end{equation*}
Similarly, we use the expression
$P_N^n = \frac{1 - \tau}{\tau}\frac{C_T^n}{C_N^n}$ to show that
\begin{equation*}
  C_T^n + P_N^n C_N^n = \frac{C_T^n}{\tau} = P^n C^n.
\end{equation*}

\subsection{Log-linearized System of
  Equations \label{Appendix_Loglinear}}

This appendix derives the log-linearized first-order conditions. To
reiterate, we log-linearize around the deterministic solution --- the
point at which the variances of shocks are zero
$\left( \sigma_{N, n} = 0 \right)$ and all firms have a capital stock
fixed at the deterministic steady-state level.

We have shown in Appendix \ref{Appendix_ModelDetails} that the
stochastic discount factor $\Lambda^n_T(\omega) / \Lambda^n_{T, 1}$ is
equalized across all households in all states. It is convenient to
write the logarithm of this stochastic discount factor as:
\begin{equation*}
  q = \lambda^n_T - \lambda^n_{T, 1}.
\end{equation*}
We can then write the log-linear first-order conditions for the second
period as
\begin{align*} (1 - \gamma) \left( \tau c_T^n + (1 - \tau) c_N^n
  \right) - c_T^n + \log \tau
  & = z^n  + q + \lambda^n_{T, 1} \\
  (1 - \gamma) \left( \tau c_T^n + (1 - \tau) c_N^n \right) - c_N^n +
  \log(1 - \tau) &= p_N^n + q + \lambda^n_{T, 1},
\end{align*}
and the log-linear resource constraints are:
\begin{align*}
  c_N^n & =  y_N^n \\
  \sum_{n = m, t, o} \theta^n c_T^n  & = 0 
\end{align*}
where $z^m = \log \left( Z^m(\omega) \right)$ and \(z^t=z^o=0\). Note
that $\Delta Res$ is a second-order term (linear in $\sigma^n$) and
consequently does not show up in the log-linear resource constraint.
This set of ten linear equations (two first order conditions for each
country and four resource constraints) allows us to solve for the
endogenous variables
$\left\{c_N^n, c_T^n, p_N^n \right\}_{n = m, t, o}$ and $q$. Keeping
in mind the log-linear relationship between each country's output and
its respective productivity shock (\ref{eqn:prodN}), it is convenient
to solve for these endogenous variables in terms of each country's
output $\left\{ y_N^m, y_N^t, y_N^o\right\}$, and the Lagrange
multipliers
$\left\{\lambda_{T, 1}^m, \lambda_{T, 1}^t, \lambda_{T, 1}^o
\right\}$.

Solving the system yields:
\begin{align*}
  c_T^m
  & = \frac{(\gamma - 1)(1 - \tau)}{1 + (\gamma - 1) \tau} \left( \bar{y}_N - y_N^m \right)
    - \frac{1 - \theta^m}{1 + (\gamma - 1) \tau} z^m
    + \frac{\bar{\lambda}_{T, 1} - \lambda^m_{T, 1}}{1 + (\gamma - 1) \tau} \\
  c_T^t
  & = \frac{(\gamma - 1)(1 - \tau)}{1 + (\gamma - 1) \tau} \left( \bar{y}_N - y_N^t \right)
    + \frac{\theta^m}{1 + (\gamma - 1) \tau} z^{m}
    + \frac{\bar{\lambda}_{T, 1} - \lambda^t_{T, 1}}{1 + (\gamma - 1) \tau}  \\
  c_T^o
  & = \frac{(\gamma - 1)(1 - \tau)}{1 + (\gamma - 1) \tau} \left( \bar{y}_N - y_N^o \right)
    + \frac{\theta^m}{1 + (\gamma - 1) \tau} z^{m}
    + \frac{\bar{\lambda}_{T, 1} - \lambda^o_{T, 1}}{1 + (\gamma - 1) \tau} \\
  c_N^n
  & = y_N^n \quad \forall n
\end{align*}
where $\bar{\lambda}_{T, 1} = \sum_n \theta^n \lambda^n_{T, 1}$. In
addition, we have the equilibrium prices:
\begin{align*}
  q
  & = - (1 - \tau)(\gamma - 1) \bar{y}_N - \theta^m z -
    \bar{\lambda}_{T, 1} \\
  p_N^m
  & = \frac{(\gamma - 1)(1 - \tau)}{1 + (\gamma - 1) \tau}
    \bar{y}_N - \frac{\gamma}{1 + (\gamma - 1) \tau} y_N^m +
    \frac{\theta^m + (\gamma - 1) \tau}{1 + (\gamma - 1) \tau}
    z^{m} + \frac{\bar{\lambda}_{T, 1} - \lambda^m_{T, 1}}{1 +
    (\gamma - 1) \tau}
    + \log [\frac{1 - \tau}{\tau}] \\
  p_N^t
  & = \frac{(\gamma - 1)(1 - \tau)}{1 + (\gamma - 1) \tau}
    \bar{y}_N - \frac{\gamma}{1 + (\gamma - 1) \tau} y_N^t +
    \frac{\theta^m}{1 + (\gamma - 1) \tau} z^{m} +
    \frac{\bar{\lambda}_{T, 1} - \lambda^t_{T, 1}}{1 + (\gamma -
    1) \tau}
    + \log [\frac{1 - \tau}{\tau}] \\
  p_N^o
  & = \frac{(\gamma - 1)(1 - \tau)}{1 + (\gamma - 1) \tau}
    \bar{y}_N - \frac{\gamma}{1 + (\gamma - 1) \tau} y_N^o +
    \frac{\theta^m}{1 + (\gamma - 1) \tau} z^{m} +
    \frac{\bar{\lambda}_{T, 1} - \lambda^o_{T, 1}}{1 + (\gamma -
    1) \tau} + \log [\frac{1 - \tau}{\tau}]
\end{align*}
It is also useful to keep track of the shadow prices of total
consumption in each country. To this end, we also use the log-linear
expression
$\lambda^n = -\gamma \left( \tau c_T^n + (1 - \tau) c_N^n \right)$
along with the solutions for traded and nontraded consumption above to
obtain:
\begin{align*}
  \lambda^m & = - \frac{(\gamma - 1)(1 - \tau) \gamma \tau}{1 + (\gamma - 1) \tau} \overline{y}_N
              - \frac{\gamma (1 - \tau)}{1 + (\gamma - 1) \tau} y_N^m
              + \frac{\left( 1 - \theta^m \right) \gamma \tau}{1 + (\gamma - 1) \tau} z^{m}
              - \frac{\gamma \tau \left( \bar{\lambda}_{T, 1} - \lambda^m_{T, 1} \right)}{1 + (\gamma - 1) \tau}\\
  \lambda^t & = - \frac{(\gamma - 1)(1 - \tau) \gamma \tau}{1 + (\gamma - 1) \tau} \overline{y}_N
              - \frac{\gamma (1 - \tau)}{1 + (\gamma - 1) \tau} y_N^t
              + \frac{\theta^m \gamma \tau}{1 + (\gamma - 1) \tau} z^{m}
              - \frac{\gamma \tau \left( \bar{\lambda}_{T, 1} - \lambda^t_{T, 1} \right)}{1 + (\gamma - 1) \tau}  \\
  \lambda^o & = - \frac{(\gamma - 1)(1 - \tau) \gamma \tau}{1 + (\gamma - 1) \tau} \overline{y}_N
              - \frac{\gamma (1 - \tau)}{1 + (\gamma - 1) \tau} y_N^o + 
              \frac{\theta^m \gamma \tau}{1 + (\gamma - 1) \tau} z^{m}
              - \frac{\gamma \tau \left( \bar{\lambda}_{T, 1} - \lambda^o_{T, 1} \right)}{1 + (\gamma - 1) \tau} 
\end{align*}

Finally, the following log-linear equations determine the first-period
Lagrange multipliers. Again, recall the stabilizing country's
government uses $\bar{Z}^m$ to add and subtract resources from the
economy to achieve (P2), which equalizes the marginal utility of
initial wealth across households. As a result:
\begin{equation*}
  \lambda^m_{T, 1} = \lambda^t_{T, 1} = \lambda^o_{T, 1} = \lambda_{T, 1},
\end{equation*}
so that the remaining endogenous term drop out of the solutions above,
\(\bar{\lambda}_{T, 1} - \lambda^m_{T, 1}=0\). Next, we normalize
$\lambda_{T, 1}$ using the second-order approximation of equation
\eqref{eqn:norm_lambda}:
\begin{equation*}
  \lambda_{T, 1}
  = \mathbb{E} \left[ \lambda_{T} \right]
  + \frac{1}{2} var\left[ \lambda_{T} \right].
\end{equation*}

\subsection{Equilibrium Asset Portfolio
  \label{Appendix_Decentralization}}

In the log-linear solution, all prices and quantities are a linear
combination of $\left\{ y_N^m, y_N^t, y_N^o\right\}$. In particular, household
expenditure, $p^n + c^n$, in each state of the world is a linear
combination of $\left\{ y_N^m, y_N^t, y_N^o\right\}$. All asset payoffs are also
linear combinations of $\left\{ y_N^m, y_N^t, y_N^o\right\}$. Any set of assets with
the same rank as the set of household expenditures will thus be able to span the
space of household expenditure. Therefore, given the appropriate set of
assets, we can write household expenditure in each state of the world as a
linear combination of these assets.

It is straightforward to verify that the set of log-linear stock
payoffs spans the space of log-linear household wealth.
\begin{lemma}
  Households in the freely floating exchange rate equilibrium hold
  levered positions in their own country's stocks and hold short
  positions in other countries' stocks,
  \begin{equation*}
    A_n^n = \frac{1 - \theta^n \tau}{1 - \tau} \text{ and }
    A_l^n = - \frac{\theta^n \tau}{1 - \tau} \text{ for } l \neq n
  \end{equation*}
  \label{Lemma_AssetPortfolio}
\end{lemma}
\begin{proof}
  The household budget constraint \eqref{eqn:budget_cons_2} can be
  re-written as:
  \begin{equation*}
    P^n(\omega) C^n(\omega)
    = \sum_{l = m, t, o} A_l^n P^n_N(\omega) Y^n_N(\omega) + Y^n_T
  \end{equation*}
  The log-linear approximation of household expenditure of the
  left-hand side is:
  \begin{equation*}
    \frac{1}{\tau} \left( p^n + c^n + \log \left[ \tau \right] \right)
    = \frac{(\gamma - 1)(1 - \tau)}{\tau \left( 1 - (\gamma - 1) \tau \right)}
    \left( \bar{y}_N - y^n_N \right) 
  \end{equation*}
  The log-linear approximation the stock portfolio payoff (right-hand
  side) is:
  \begin{equation*}
    \sum_{l = m, t, o} A_l^n \frac{1 - \tau}{\tau}
    \left( p^{l \ast}_N + y^l_N - \log\left[ \frac{1 - \tau}{\tau} \right] \right).
  \end{equation*}
  where:
  \begin{equation*}
    p^{l \ast}_N + y^l_N
    = \frac{(\gamma - 1)(1 - \tau)}{1 + (\gamma - 1) \tau }
    \left( \bar{y}_N - y^l_N \right) 
    + \log\left[ \frac{1 - \tau}{\tau} \right].
  \end{equation*}
  We equate household expenditures in each state of the world with the
  portfolio payoff:
  \begin{equation*}
    \frac{(\gamma - 1)(1 - \tau)}{\left( 1 - (\gamma - 1) \tau \right)}
    \left( \bar{y}_N - y^n_N \right) 
    = \sum_{l = m, t, o} A_l^n \frac{1 - \tau}{\tau}
    \left( p^{l \ast}_N + y^l_N - \log\left[ \frac{1 - \tau}{\tau} \right] \right)
  \end{equation*}
  Since this equation holds state-by-state, we solve for the shares,
  $A^n_l$, by matching the coefficients on $y^l_N$ in the portfolio
  payoff with the coefficients on $y^l_N$ in household expenditure.
\end{proof}


\subsection{Derivation of Equation
  \eqref{eq_link_k_r} \label{app_cap}}

To derive (\ref{eq_link_k_r}), we use the following second-order
approximation of equation (\ref{eqn:FOCK}):
\begin{equation*}
  \lambda_{T, 1} + q_K + k^n
  = \log[\nu] + \mathbb{E}\left[ \lambda_N^n + y_N^n \right]
  + \frac{1}{2} \text{var}\left( \lambda_N^n + y_N^n \right),
\end{equation*}
%$k^n = 0$, $\mathbb{E}[y_N^n] = -\frac{1}{2} \sigma^2_N$ , and $\text{var}[y_N^n] = \sigma_N^2$.
Next, we substitute
$\lambda^{n}_{N}=p^{n}_{N}+ \lambda_T$, and take differences across
two arbitrary countries $f$ and $h$ to obtain:
\begin{equation}
  k^f - k^h
  = \frac{1}{2} \text{var}\left( p_{N}^f + y_{N}^f \right)
  - \frac{1}{2} \text{var}\left( p_{N}^h + y_{N}^h \right)
  + \text{cov}\left( p_{N}^f + y_{N}^f - p_{N}^h - y_{N}^h, \lambda_{T}\right).
  \label{eqn:kspread} 
\end{equation}
For any country $n$:
\begin{equation*}
  p_{N}^{n*}  + y_{N}^{n*} = 
  \frac{(1 - \tau)(\gamma - 1)}{1 + (\gamma - 1) \tau}
  \left( \bar{y}_{N} - y_{N}^n \right).
\end{equation*}
Plugging this expression for $p_{N}^{n*} + y_{N}^{n*}$ into the
right-hand side of equation (\ref{eqn:kspread}) shows:
\begin{equation*}
  k^{f*} - k^{h*} = 
  \frac{(\gamma - 1)^3 (1 - \tau)^2 \tau}{1 + (\gamma - 1) \tau}
  \left( \theta^f - \theta^h \right)\sigma_N^2. 
\end{equation*}
Combine this equation with equation (\ref{eq_FF_UIP}) to derive
(\ref{eq_link_k_r}),

\subsection{Proof of Lemma
  \ref{lemma:CostofPeg}
  \label{Appendix_ProofCostofPeg}}

First, we solve for the state-contingent taxes that implement the real
exchange rate stabilization. Afterwards, we derive an expression for
the cost of stabilizing the exchange rate. We guess a tax of the form
$Z(\omega) = \left( Y_N^m(\omega) / Y_N^t(\omega) \right)^a$
stabilizes the exchange rate for a constant $a$ and solve for the
coefficient $a$ that stabilizes the real exchange rate. In logs, this
tax is: $z = a \left( y_N^t - y_N^m \right)$. We plug this expression
into the solution of the model derived in Appendix
\ref{Appendix_Loglinear} and solve for the log real exchange rate:
\begin{equation*}
  s^{m, t} = 
  \frac{\gamma (1 - \tau)}{1 + (\gamma - 1) \tau} \left( y_N^t - y_N^m \right) 
  + a \frac{\gamma \tau}{1 + (\gamma - 1) \tau} \left( y_N^m - y_N^t \right). 
\end{equation*}
Choose $a$ such that $s^{m, t} = (1 - \zeta) s^{m, t \ast}$. This
yields $a = \zeta (1 - \tau) / \tau$. Finally, we use the expression
for $s^{f, h \ast}$ given by equation \eqref{eqn:rerNP} to write $z$
as a function of $p^{m \ast}$ and $p^{t \ast}$.

$\Delta Res$ is defined by equation (\ref{eqn:kappacost}). First, we
solve for $\bar{Z}$ by plugging in the equilibrium consumption of
nontraded goods and $K^m = 1$ into the budget constraint
\eqref{eqn:budget_cons_1}:
\begin{equation*}
  \bar{Z}
  = \left( A^m_m - 1 \right) Q^m_N + \sum_{l \neq m} A^m_l Q^l_N 
  - \kappa^m.
\end{equation*}
Next, multiply equation (\ref{eqn:budget_cons_2}) by the stochastic
discount factor, $\Lambda_T(\omega) / \Lambda_{T, 1}$, and take
expectations to derive the present value of tax revenues:
\begin{align*}
  &  \mathbb{E}\left[ 
    \left( \frac{\Lambda_T(\omega)}{\Lambda_{T, 1}} \right) 
    \left(Z(\omega) - 1\right) C^m_T(\omega) \right] \\
  = & \mathbb{E}\left[ 
      \left( \frac{\Lambda_T(\omega)}{\Lambda_{T, 1}} \right) 
      \left( \left( A^m_m - 1 \right) P^m_N(\omega) Y^m_N(\omega) +
      \sum_{l \neq m} A^m_l P^m_N(\omega) Y^m_N(\omega) + Y^n_T -
      C^m_T(\omega) \right) \right] \\
  = & \left( A^m_m - 1 \right) Q^m_N + \sum_{l \neq m} A^m_l Q^l_N + 
      Y^m_T \mathbb{E} \left[ \left( 
      \frac{\Lambda_T(\omega)}{\Lambda_{T, 1}} \right) \right] - 
      \mathbb{E}\left[ 
      \left( \frac{\Lambda_T(\omega)}{\Lambda_{T, 1}} \right) 
      C^m_T(\omega) \right] 
\end{align*}
Finally, we derive $\kappa^m$. In the freely floating exchange rate
economy, $\bar{Z} = 0$ and $Z(\omega) = 1$. Plug these values into
equations (\ref{eqn:budget_cons_2}) and (\ref{eqn:budget_cons_1}).
Finally, we substitute equation \eqref{eqn:norm_lambda} and use the
fact that $Y^m_T$ is a constant to show:
\begin{equation}
  \kappa^m = \mathbb{E}\left[ 
    \left( \frac{\Lambda^{\ast}_T(\omega)}{\Lambda^{\ast}_{T, 1}} \right) 
    C_T^{m \ast}(\omega) \right]  - Y^m_T 
  \label{eqn:defn_kappa}
\end{equation}
We plug the expressions for $\bar{Z}$, the present value of tax
revenues, and $\kappa^m$ in equation \eqref{eqn:kappacost}, and
simplify to arrive at equation \eqref{eq:kappacost}.

We also derive the portfolio of stocks that exactly finances the
stabilization policy. This is the portfolio that pays the difference
between traded consumption when stabilizing and traded consumption in
the freely floating regime given by equation \eqref{eqn:cTp}. For
convenience, this equation is repeated here where
$p^{t \ast} - p^{m \ast}$ is written in terms differences in nontraded
output:
\begin{equation*}
  c^m_T - c^{m \ast}_T
  = \zeta
  \frac{(1 - \theta^m)(1 - \tau)}{\tau \left( 1 + (\gamma - 1) \tau \right)}
  \left( y^t_N - y^m_N \right).
\end{equation*}
We use the same log-linear approximation of the stock portfolio as in
Appendix \ref{Appendix_Decentralization}. Letting $A^m_l$ denote the
number of shares of country $l$ stock the stabilizing country's
central bank holds, we get\begin{equation*} A^m_m = \zeta \frac{1 -
    \theta^m}{\gamma - \zeta (\gamma - 1)(1 - \tau)} \text{, } A^m_t =
  - A^m_m \text{, } A^m_o = 0.
\end{equation*}


\subsection{Proof of Proposition
  \ref{prop:KNRBCPeg} \label{Appendix_KNRBCPeg}}

We use the expressions from Appendix \ref{Appendix_Loglinear} to
calculate $p^t - p^m = \lambda^t - \lambda^m$ and we plug the
resulting expression into equation \eqref{eq_UIP_RF}:
\begin{align*}
  r^m + \Delta \mathbb{E}s^{m, t} - r^t
  &= \text{cov}\left( \lambda_T, p^t - p^m \right) \\
  &= \left(r^{m \ast} + \Delta \mathbb{E}s^{m, t \ast} - r^{t \ast} \right)
    - \zeta \frac{(1 - \tau)^2 \gamma \left( 2 \theta^m (1 - \zeta)
    + \left( \theta^t - \theta^m \right) (\gamma - 1) \tau \right)}
    {\tau \left( 1 + (\gamma - 1) \tau \right)} \sigma_N^2.
\end{align*}
When the stabilizing country is smaller than the target country,
$\theta^m < \theta^t$, the right-hand side of this expression implies
the stabilization decreases the risk-free rate in the stabilizing
country relative to the risk-free rate in the target country.

We use equation (\ref{eqn:kspread}) to calculate the differential
incentives to accumulate capital:
\begin{equation*}
  k^m - k^t = k^{m \ast} - k^{t \ast} + \zeta 
  \left( \frac{(\gamma - 1)^2 (1 - \tau)^2\left( (1 - 2 \theta^m)(1 - \zeta) + (\theta^t - \theta^m)(\gamma - 1) \tau \right)}{\left( 1 + (\gamma - 1) \tau \right)^2 } \right) \sigma_N^2.
\end{equation*}
The last term of the right-hand side of this expression shows that
incentives to accumulate capital in the stabilizing country increase
relative to the target country as long as
\begin{equation*}
  \theta^t > \theta^m + \frac{(1 - 2 \theta^m)(1 - \zeta)}{\tau (\gamma - 1)}. 
\end{equation*}
Because firms are competitive, wages are given by the marginal product
of labor.
$w^n = (1 - \nu) \exp\left( \eta^n \right) \left(K^n\right)^\nu $.
Since the marginal product of labor rises with the level of capital
accumulation, the exchange rate stabilization increases wages in the
stabilizing country relative to all other countries.

Recall, the world-market value of the country $m$ domestic firm given
by equation \eqref{eqn:FOC_Stock} is:
\begin{equation*}
  Q^m_N = \mathbb{E}\left[
    \frac{\Lambda_T(\omega)}{\Lambda_{T, 1}} P^m_N Y^m_N
  \right].
\end{equation*}
The second-order log-linear approximation of the world-market value of
the country $m$ domestic firm is:
\begin{equation*}
  q^m_N 
  = \mathbb{E} \left[ \lambda_T - \lambda_{T, 1} + p^m_N + y^m_N \right] 
  + \frac{1}{2} var\left[ \lambda_T - \lambda_{T, 1} + p^m_N + y^m_N \right].
\end{equation*}
The spread between the value of the firm in the stabilizing and target
countries yields the same expression as the right-hand side of
equation \eqref{eqn:kspread}. Hence, we have already shown the value
of the firm in the stabilizing country increases relative to the
target country if $\theta^t$ is large enough.


\subsection{Proof of Proposition
  \ref{prop:CostSmallP} \label{Appendix_CostSmallP}}

Equation (\ref{eq:kappacost}) shows the cost of the stabilization is
the difference in the value of traded consumption between the freely
floating regime and stabilized regime. We derive a second-order
log-linear approximation of the value of traded consumption:
\begin{equation*}
  v^m_{T} 
  = \mathbb{E}\left[ \lambda_T - \lambda_{T, 1} + c_T^m \right]
  + \frac{1}{2} \text{var} \left[ \lambda_T - \lambda_{T, 1} + c_T^m \right].
\end{equation*}
We plug the expressions for $\lambda_T$, $\lambda_{T, 1}$, and $c^m_T$
into the previous equation in order to derive the change in the log
value of traded consumption
\begin{equation*}
  v^m_T - v_T^{m \ast} = 
  \frac{\left(\left( \zeta + (\gamma - 1) \tau \right) - 
      \tau^2 (1 - \gamma)^2 \theta^t \right) (1 - \tau)^2 
    \zeta \sigma_N^2}{\tau^2 \left( 1 + (\gamma - 1)\tau \right)^2}.
\end{equation*}
This expression is decreasing in the size of the target country, and
becomes negative if and only if the target country is large enough:
$\theta^t > \left( \zeta + (\gamma - 1) \tau \right) / \left( \tau
  \left( \gamma - 1 \right) \right)^2$.

Next, we evaluate the derivative of $v^m_T - v_T^{m \ast}$ with
respect to $\theta^m$ at the point where $\theta^m = 0$:
\begin{equation*}
  \left .
    \frac{\partial (v^m_T - v_T^{m \ast})}{\partial \theta^m}
  \right \vert_{\theta^m = 0}
  = \zeta
  \frac{(\gamma - 1)(1 - \tau)^2\left( \theta^t + 2 \zeta + 2 (1 + \theta^t) (\gamma - 1) \tau \right)}{\tau \left( 1 + (\gamma - 1) \tau \right)^2} \sigma_N^2 > 0
\end{equation*}
Hence, the cost of the stabilization increases locally with the size
of the stabilizing country.


\subsection{Proof of Proposition
  \ref{prop:KNRBCTarget} \label{Appendix_KNRBCTarget}}

We use the expressions from Appendix \ref{Appendix_Loglinear} to
calculate $p^o - p^t = \lambda^o - \lambda^t$ and we plug the resulting
expression into equation \eqref{eq_UIP_RF}:
\begin{equation*}
  r^t + \Delta \mathbb{E}s^{t, o} - r^o
  = \text{cov}\left( \lambda_T, p^o - p^t \right)
  = \left(r^{t \ast} + \Delta \mathbb{E}s^{t, o \ast} - r^{o \ast} \right)
  + \zeta \frac{\theta^m (1 - \tau)^2 \gamma}{\tau \left( 1 + (\gamma - 1) \tau \right)} \sigma_N^2,
\end{equation*}
which implies the exchange rate stabilization increases the risk-free
rate in the target country relative to the risk-free rate in the
outside country. 

We use equation (\ref{eqn:kspread}) to calculate the differential
incentives to accumulate capital,
\begin{equation*}
  k^t - k^o = k^{t \ast} - k^{o \ast} -
  \frac{\theta^m (\gamma - 1)^2 (1 - \tau)^2}{\left( 1 + (\gamma - 1) \tau \right)^2} \zeta \sigma_N^2
\end{equation*}
The last term on the right-hand side shows that incentives to
accumulate capital in the target country decrease relative to the
outside country.

Because firms are competitive, wages are given by the marginal product
of labor. Since the marginal product of labor rises with the level of
capital accumulation, the exchange rate stabilization decreases wages
in the target country relative to all other countries.

Finally, we show that if the stabilizing country is smaller than the
target country, $\theta^m < \theta^t$, then the stabilization lowers
the volatility of consumption in the target country. The log-linear
approximation of household consumption in the target country is,
$c^t = \tau c_T^t + (1 - \tau) c_N^t$. We use the expression for
traded consumption derived in Appendix \ref{Appendix_Loglinear} and
the expression for the state-contingent tax derived in Appendix
\ref{Appendix_ProofCostofPeg} to derive the volatility of aggregate
consumption in the target country:
\begin{equation*}
  \text{var} \left( c^t \right)
  = \text{var} \left( c^{t \ast} \right) 
  - \zeta \frac{2 \theta^m (1 - \tau)^2 \left( 1 - \theta^m \zeta + \left( \theta^t - \theta^m \right)(\gamma - 1) \tau \right)}{\left( 1 + (\gamma - 1) \tau \right)^2}\sigma_N^2.
\end{equation*}
Therefore, $\text{var} \left( c^t \right)$ decreases when a country
stabilizes its exchange rate relative to the target country as long as
the stabilizing country is smaller, $\theta^t > \theta^m$.

\section{Appendix to Section \ref{sec_welfare}
  \label{Appendix_Welfare}}

Households continue to maximize utility subject to their budget
constraints \eqref{eqn:budget_cons_2} and \eqref{eqn:budget_cons_1}.
However, when $\Delta Res = 0$ and households hold the portfolio of
assets derived in Appendix \ref{Appendix_Decentralization}, their
initial wealth is:
\begin{equation*}
  W^n_0
  = \sum_{l \in \left\{ m, t, o \right\}} A^{n \ast}_l Q^l_N
  + Q_K K^{n \ast}_N + \bar{Z}. 
\end{equation*}
Because $\Delta Res = 0$,
\begin{equation*}
  \bar{Z} = \mathbb{E}\left[
    \frac{\Lambda^n_T(\omega)}{\Lambda^n_{T, 1}}
    \left( Z^n(\omega) - 1 \right) C^n_T(\omega)
  \right] 
\end{equation*}
is just the present value of tax-revenues. To re-iterate,
$A^{n \ast}_l$ denotes the country $n$ households holding of the
country $l$ stock in the freely floating exchange rate regime. We plug
this value of $W^n_0$ into the budget constraint
\eqref{eqn:budget_cons_1}:
\begin{equation*}
  \sum_{l \in \left\{ m, t, o \right\}} A^n_l Q^l_N
  + Q_K K^{n \ast}_N
  = \sum_{l \in \left\{ m, t, o \right\}} A^{n \ast}_l Q^l_N
  + Q_K K^{n \ast}_N + \bar{Z}
\end{equation*}
Next, we multiply equation \eqref{eqn:budget_cons_2} by the stochastic
discount factor and take expectations:
\begin{align*}
  \mathbb{E}\left[
  \frac{\Lambda^n_T(\omega)}{\Lambda^n_{T, 1}}
  \left( Z^n(\omega) C^n_T(\omega) 
  + P^n_N(\omega) C^n_N(\omega) \right) \right]
  & = \sum_l \mathbb{E}\left[
    \frac{\Lambda^n_T(\omega)}{\Lambda^n_{T, 1}}
    \left( A^n_l P^l_N(\omega) Y^l_N(\omega)
    + Y_T \right) \right] \\
  & = \sum_{l \in \{m, t, o\}} A^n_l  Q^l_N
    + \mathbb{E}\left[
    \frac{\Lambda^n_T(\omega)}{\Lambda^n_{T, 1}} Y_T
    \right] 
\end{align*}
We subtract the two equations from each other and re-arrange:
\begin{equation*}
  \mathbb{E}\left[
    \frac{\Lambda^n_T(\omega)}{\Lambda^n_{T, 1}}
    \left( Z^n(\omega) C^n_T(\omega) 
      + P^n_N(\omega) C^n_N(\omega) \right) \right]
  = \sum_{l \in \left\{ m, t, o \right\}} A^{n \ast}_l Q^l_N
  + \bar{Z} + \mathbb{E}\left[
    \frac{\Lambda^n_T(\omega)}{\Lambda^n_{T, 1}} Y_T
  \right]
\end{equation*}
Because households must consume their endowment of non-traded goods,
we subtract the value of nontraded consumption from both sides. We
cancel out the present value of tax revenues with the lump sum
transfer $\bar{Z}$. Finally, we apply equation \eqref{eqn:norm_lambda}
to arrive at the following expression for the value of traded
consumption in country $n$:
\begin{equation}
  \mathbb{E}\left[
    \frac{\Lambda^n_T(\omega)}{\Lambda^n_{T, 1}} C^n_T(\omega) 
  \right]
  = \left( A^{n \ast}_n - 1 \right) Q^n_N
  + \sum_{l \neq n} A^{n \ast}_l Q^l_N + Y^n_T.
  \label{eqn:bc_stocks}
\end{equation}
The left-hand side represents the value of traded consumption when
stabilizing. The right-hand side represents the household's wealth
from its portfolio of stocks after subtracting out expenditure on
nontraded consumption and capital investment.

Next, we derive a second-order approximation for equation
\eqref{eqn:bc_stocks}:
\begin{equation}
  \mathbb{E} \left[ \lambda_T - \lambda_{T, 1} + c^n_T \right] +
  \frac{1}{2} \text {var} \left[ \lambda_T - \lambda_{T, 1} + c^n_T
  \right] = \frac{1-\tau}{\tau} \left( \left( A^{n \ast}_n - 1 \right)
    q^n_N + \sum_{l \neq n}A^{l \ast}_t q^t_N \right) + q_T
  \label{eqn:bc1_peg_nores}
\end{equation}
where:
\begin{equation*}
  q_N^n = \mathbb{E} \left[ \lambda_T - \lambda_{T, 1} + p^n_N + y^n_N \right]
  + \frac{1}{2} \text{var} \left[ \lambda_T - \lambda_{T, 1} + p^n_N + y^n_N \right]
\end{equation*}
and
\begin{equation*}
  q_T^n = 0,
\end{equation*}
In keeping with the solution method in Section \ref{sec:real}, we
solve for the equilibrium valuation change in households' portfolios
using a second-order approximation around the point at which the
marginal utility of wealth of households in all countries is
equalized. Hence, we plug in the expressions for $p^n_N$, $y^n_N$,
$\lambda_{T}$ and , $\lambda_{T, 1}$ from Section \ref{sec:real} into
the expressions for $q^n_N$, which appear on the right-hand side of
equation \eqref{eqn:bc1_peg_nores}. These expressions are given by
equations \eqref{eqn:pN_yN}, \eqref{eqn:lambdat2} and
\eqref{eqn:norm_lambda}.

We solve for the Lagrange multiplier $\lambda^n_{T, 1}$ (from the
left-hand side of equation \eqref{eqn:bc1_peg_nores}) that satisfies
equation \eqref{eqn:bc1_peg_nores}. We let $\lambda^n_{T, 1, Stock}$
denote the set of Lagrange multipliers derived from solving equation
\eqref{eqn:bc1_peg_nores}. After solving the Lagrange multipliers, we
obtain solutions for traded consumption by plugging the Lagrange
multipliers, $\lambda^n_{T, 1, Stock}$, into the log-linear
expressions for $c^n_T$ derived in Appendix \ref{Appendix_Loglinear}.
These new expressions for traded consumption reflect the level shifts
in traded consumption due to changes in the value of the household's
stock portfolio.

We calculate changes in welfare using a second-order approximation of
household utility:
\begin{equation}
  u^n = \frac{1}{1 - \gamma}\log \left[ (1 - \gamma) U^n \right] = 
  \mathbb{E}[c^n] - \frac{\gamma - 1}{2} \text{var} [c^n]
  \label{eqn:utility_approx}
\end{equation}
where $c^n = \tau c^n_T + (1 - \tau) c^n_N$. We plug in the solutions
for $c^n_T$, with the Lagrange multipliers derived above, into the
welfare function. Define the welfare change
$\Delta u^n = u^n - u^{n \ast}$, where $u^{n \ast}$ is the value of
$u^n$ when $\zeta = 0$. The welfare change in the stabilizing country
is:
\begin{equation*}
  \begin{split}
    \Delta u^m & = \frac{ \zeta (\gamma -1)^2 (\theta^m - 1) (\tau
      -1)^2 \tau ((\gamma -1) \tau (\theta^m - \theta^t) + 1)
    }{(1 + (\gamma -1) \tau)^2} \sigma_N^2 \\
    & + \frac{ \zeta^2 (1 - \tau)^2 \left( (\gamma -1) \left(\left(
            \theta^m \right)^2 - 1\right) \tau + (\gamma - 1)^2
        (\theta^m - 1) (2 \theta^m - 1) \tau^2 + \theta^m - 1 \right)
    }{\tau (1 + (\gamma - 1) \tau)^2} \sigma_N^2
  \end{split}
\end{equation*}
Equation \eqref{eq:Deltau} displays the welfare consequences for a
small stabilizing country $\left( \theta^m = 0 \right)$.

In order to decompose changes in welfare, we also compute traded
consumption under the assumption there is no valuation effect. In this
calculation, the household's value of traded consumption post exchange
rate stabilization exactly equals the value of traded consumption
prior to the stabilization:
\begin{equation}
  \mathbb{E} \left[ \lambda_T - \lambda_{T, 1} + c^n_T \right] +
  \frac{1}{2} \text {var} \left[ \lambda_T - \lambda_{T, 1} + c^n_T
  \right]
  = \mathbb{E} \left[
    \lambda^{\ast}_T - \lambda^{\ast}_{T, 1} + c^{n \ast}_T
  \right]
  + \frac{1}{2} \text {var} \left[
    \lambda^{\ast}_T - \lambda^{\ast}_{T, 1} + c^{n \ast}_T
  \right].
  \label{eqn:bc1_peg_AD}
\end{equation}
Denote the Lagrange multipliers derived from solving equation
\eqref{eqn:bc1_peg_AD} by $\lambda^n_{T, 1, AD}$. Again, we plug these
the Lagrange multipliers $\lambda^n_{T, 1, AD}$ into the expressions
from Appendix \ref{Appendix_Loglinear} to derive expressions for
traded consumption with stabilization, but without any change in the
total value of traded consumption.

The first term of \eqref{eq:Deltau} is calculated by plugging the
expression for $c^n_T$ with the Lagrange multipliers
$\Lambda^n_{T, 1, AD}$ into $c^n$ and deriving the change in
$\mathbb{E}\left[ c^n \right]$ when $\zeta$ deviates from zero. The
$\Delta \text{var}\left[ c^m \right]$ term reflects the change
captured by $\frac{\gamma - 1}{2}\text{var}\left[ c^m \right]$. The
``Valuation Effect'' is calculated by plugging the expression for
$c^n_T$ with the Lagrange multipliers $\Lambda^n_{T, 1, Stock}$ into
$c^n$, deriving the change in $\mathbb{E}\left[ c^n \right]$ when
$\zeta$ deviates from zero, and then subtracting out the first term of
\eqref{eq:Deltau}.

When the first term of \eqref{eq:Deltau} is combined with the
``Valuation Effect'':
\begin{equation*}
  \Delta u^m
  = - \frac{\zeta^2 (1 - \tau)^2 }{
    \tau \left( 1 + (\gamma - 1) \tau\right)} \sigma_N^2
  + \frac{ \left( \zeta \Theta^t + \zeta^2 \right)
    \theta^t \tau (\gamma - 1)^2 (1 - \tau)^2 }{
    \left( 1 + (\gamma - 1) \tau \right)^2} \sigma_N^2
\end{equation*}
The first term on the right-hand side is clearly negative, which
indicates the welfare losses from the increase in consumption
volatility are larger than any gains from accumulating reserves.

Finally, equation \eqref{eq:Deltau} can be condensed to:
\begin{equation*}
  \Delta u^m = \zeta
  \frac{(1 - \tau)^2
    \left( - \zeta \left( 1 + (\gamma - 1) \tau \right) 
      + (\theta^t (\gamma - 1) \tau - 1 + \zeta)(\gamma - 1)^2 \tau^2 \right)}
  {\tau \left( 1 + (\gamma - 1) \tau \right)^2} \sigma_N^2.
\end{equation*}
The right-hand side of this equation is positive if:
\begin{equation*}
  \theta^t
  > \bar{\theta}
  = \frac{1 - \zeta}{(\gamma - 1) \tau}
  + \frac{\zeta (1 + (\gamma - 1) \tau)}{(\gamma - 1)^3 \tau^3}.
\end{equation*}


\subsection{Equilibrium Bond Portfolio
  \label{Appendix_BondDecentralization}}

Suppose households are confined to trading international risk-free
bonds rather than stocks. The country $n$ risk-free bond pays
$P^n(\omega)$ units of the traded good in state $\omega$ of period 2.
Similar to the exercise in Appendix \ref{Appendix_Decentralization},
these asset payoffs are linear combinations of the nontraded output in
each country. Likewise, it is straightforward to verify the set of
log-linear bond payoffs spans the space of log-linear household
wealth. As a result, equilibrium outcomes in the economy are
unaffected by the change in the asset space. We just need to solve for
the household bond portfolios that pay the appropriate payoff in each
state in the second period.

Let $B^n_l$ denote the number of country $l$ bonds purchased by
households in country $n$. Hence, the log-linear approximation of the
payoff received from the portfolio held by country $n$ households is:
\begin{equation*}
  \sum_{l = m, t, o} B_l^n \frac{1}{\tau} p^l
\end{equation*}
Again, we solve for the portfolio weights, $B^n_l$, by matching the
coefficients on $y^l_N$ in the portfolio payoff with the coefficients
on $y^l_N$ in household expenditure. This procedure yields the
following result:
\begin{equation*}
  B_n^n = \frac{(1 - \theta^n)(\gamma - 1)}{\gamma} \text{ and }
  B_l^n = - \frac{\theta^l (\gamma - 1)}{\gamma} \text{ for } l \neq n.
\end{equation*}

Households thus hold levered positions in their domestic risk-free
bond. Proposition \ref{prop:KNRBCPeg} shows the stabilizing country's
risk-free rate decreases when the target country is larger than the
stabilizing country, increasing the relative value of its bonds. As a
result, the same intuition from Proposition \ref{prop_welfare} shows
that announcing a stabilization relative to a larger country increases
the stabilizing country's share of world wealth and thus, by the same
logic, can increase the welfare of its households.

\subsection{Welfare Consequences in Target and Outside Countries
  \label{Appendix_Welfare_TO}}

In this appendix, we provide expressions for the welfare consequences
of stabilization on households in the target and outside countries.
Analogous to the calculation of $\Delta u^m$, we plug the Lagrange
multipliers derived in Appendix \ref{Appendix_Welfare} into the
expression of $c^t_T$ and $c^o_T$ derived in Appendix
\ref{Appendix_Loglinear}. We again plug the value of $c^t_T$ into the
second-order approximation of household welfare given by equation
\eqref{eqn:utility_approx}:
\begin{equation*}
  \begin{split}
    \Delta u^t & = \frac{\zeta^2 \theta^m (1 - \tau)^2 ((\gamma -1)
      \tau ((\gamma -1) (2 \theta^m - 1) \tau
      + \theta^m) + 1)}{\tau  (1 + (\gamma -1) \tau)^2} \sigma^2_N \\
    & + \frac{(\gamma -1) \zeta \theta^m (1 - \tau)^2 ((\gamma -1)
      \tau ((\gamma -1) \tau (\theta^m - \theta^t) + 1) + 2)}{((\gamma
      -1) \tau + 1)^2} \sigma^2_N.
  \end{split}
\end{equation*}
The analogous calculation for the outside country yields:
\begin{equation*} \Delta u^o = \Delta u^t - \frac{\zeta \theta^m
    (\gamma - 1)(1 - \tau)^2} {\left( 1 + (\gamma - 1) \tau\right)^2}
  \sigma_N^2.
\end{equation*}
Households in the outside country are weakly worse off than households
in the target country as a result of the stabilization.

\section{Appendix to Section \ref{sec:nominal}}\label{Appendix_Nominal_Rigidities}
  

\subsection{Sticky Prices of Traded Goods
  \label{Appendix_RigidPrices}}

In this appendix, we provide additional details about the model setup,
derive the monetary policy rule that implements an exchange rate
stabilization, and afterwards, relate the seigniorage from
stabilization to $-\Delta Res$.

Households enter each period with a fixed quantity of domestic
currency, and all goods consumed in a given country must be purchased
using this domestic currency. In the first period, households use
their currency to purchase stocks. The first period cash-in-advance
constraint reads:
\begin{equation*}
  \tilde{P}^n_T \left( \sum_l A^n_l Q^l_N + Q_K K^n_N \right) \le \tilde M^n_1.
\end{equation*}
where $\tilde M^n_1$ is the quantity of currency available to country
$n$ households in period 1.

Households enter the first period with the monetary value of the claim
to their firm, $Q_N$, their endowment of capital, $Q_k$, and their
transfer, $\kappa^n$, as well as any additional money balances the
central bank injects through open market operations, $\Delta M^n_1$.
The household's first-period budget constraint can thus be expressed
as:
\begin{equation}
  \tilde{P}^n_T \left( \sum_l A^n_l Q^l_N + Q_K K^n_N \right)
  \le \tilde{P}^n_T \left( Q^n_N + Q_K + \kappa^n \right) + \Delta M^n_1.
  \label{eqn:bc_nom_1}
\end{equation}
Hence,
$\tilde M^n_1 = \tilde{P}^n_T \left( Q^n_N + Q_K + \kappa^n \right) +
\Delta M^n_1$.

In the second period, households face the cash-in-advance constraint:
\begin{equation*}
  \tilde{P}^n_T C^n_T(\omega) + \tilde{P}^n_N(\omega) C^n_N(\omega)
  \le \tilde M_2^n(\omega),
\end{equation*}
where $\tilde M_2^n(\omega)$ is the total quantity of currency
available to country $n$ households in period 2 to purchase
consumption. Households again enter the second period with the
monetary payoff from their stock portfolio and any additional money
balances the central bank injects through open market operations:
\begin{equation*}
  \Delta \tilde M^n(\omega)
  = \tilde{M}^n_2(\omega) - \tilde{P}^n_T
  \left( \sum_l A^n_l P^l_N(\omega)Y^l_N(\omega) + Y^n_T \right).
\end{equation*}
As a result, the household's budget constraint in the second period
can be expressed as:
\begin{equation}
  \tilde{P}^n_T C^n_T(\omega)
  + \tilde{P}^n_N(\omega) C^n_N(\omega)
  \le  \tilde{P}^n_T
  \left( \sum_l A^n_l P^l_N(\omega)Y^l_N(\omega) + Y^n_T \right) +
  \Delta \tilde M^n(\omega).
  \label{eqn:bc_nom_2}
\end{equation}

In the second period, households maximize utility (\ref{eqn:utility})
subject to (\ref{eqn:bc_nom_2}). Letting $\tilde{\Lambda}^n_T(\omega)$
denote the Lagrange multiplier on the household's budget constraint,
the first-order conditions with respect to traded and nontraded
consumption are:
\begin{align}
  \tau \left( C^n(\omega) \right)^{1 - \gamma} \left( C^n_T(\omega) \right)^{-1}
  & = \tilde{P}^n_T \tilde{\Lambda}^n(\omega) 
    \label{nomeqn:FOCT} \\
  (1 - \tau) \left( C^n(\omega) \right)^{1 - \gamma} \left( C^n_N(\omega) \right)^{-1}
  & = \tilde{P}^n_N(\omega) \tilde{\Lambda}^n(\omega) 
    \label{nomeqn:FOCN}
\end{align}
Because households exhibit Cobb-Douglas utility, they spend a fraction
$\tau$ of their wealth on the traded good, and a fraction $1 - \tau$
of their wealth on the nontraded good:
\begin{align}
  C^n_T(\omega)
  &= \tau \frac{\tilde{M}^n_2(\omega)}{\tilde{P}^n_T} 
    \label{nomeqn:CT_share} \\
  C^n_N(\omega)
  &= (1 - \tau) \frac{\tilde{M}^n_2(\omega)}{\tilde{P}^n_N(\omega)}. 
    \label{nomeqn:CN_share}
\end{align}
Since the price of the traded good is fixed, equation
\eqref{nomeqn:CT_share} immediately shows that monetary policy
determines the consumption of traded goods in each country. With the
consumption of the nontraded goods being determined by the capital
stock and the exogenous productivity shock, central banks are able to
neutralize the monetary friction and can therefore replicate the
freely floating allocation in the baseline model.

Next, we derive the monetary policy that stabilizes exchange rates. In
equilibrium, households must consume their country's endowment of
nontraded goods. Hence, equation \eqref{nomeqn:CN_share} shows the
nominal price of the nontraded good can be written as a function of
the money supply and the nontraded endowment:
\begin{equation}
  \tilde{P}^n_N(\omega) = (1 - \tau) \tilde{M}^n_2(\omega) / Y^n_N(\omega).
  \label{nomeqn:PN}
\end{equation}
We plug equation \eqref{nomeqn:PN} into \eqref{nomeqn:FOCN} to derive
traded consumption as a function of $\tilde{M}_2^n(\omega)$,
$Y^n_N(\omega)$ and $\tilde{\Lambda}^n(\omega)$:
\begin{equation}
  C^n_T(\omega) 
  = \tau^{\frac{1}{\tau (1 - \gamma)}} \left( 1 - \tau \right)^{\frac{1}{\tau (\gamma - 1)}}
  Y^n_N(\omega)^{\frac{1 - \tau}{\tau}} 
  \tilde{M}^n_2(\omega)^{\frac{1}{\tau (1 - \gamma)}}
  \tilde{\Lambda}^n(\omega)^{\frac{1}{\tau (1 - \gamma)}}.
  \label{nomeqn:CT}
\end{equation}
Next, we plug \eqref{nomeqn:CT} into \eqref{nomeqn:FOCT} to derive a
relationship between $\tilde{P}^n_T$, $\tilde{M}^n_2(\omega)$ and
$\tilde{\Lambda}^n(\omega)$:
\begin{equation}
  \tilde{P}^n_T =
  \tau^{2 + \frac{1}{\tau (\gamma - 1)}} (1 - \tau)^{-1 - \frac{1}{\tau (\gamma - 1)}}
  Y^n_N(\omega)^{- \frac{(1 - \tau)(1 + 2 \tau (\gamma - 1))}{\tau}}
  \tilde{M}^n_2(\omega)^{1 + \frac{1}{\tau (\gamma - 1)}}
  \tilde{\Lambda}^n(\omega)^{\frac{1}{\tau (\gamma - 1)}}.
  \label{nomeqn:PT}
\end{equation}

\begin{prop}
  The central banks of the target and outside countries float their
  exchange rates by choosing $M^n_2(\omega) $ to neutralize the
  monetary friction in their countries
  \begin{equation}
    M^n_2(\omega) = (1 - \tau) \tau^{-2 + \frac{1}{1 + \tau (\gamma - 1)}}
    Y^n_N(\omega)^{-1 + 2 \tau + \gamma \left( 2 (1 - \tau) - \frac{1}{1 + \tau (\gamma - 1)} \right)}
    \left(\Lambda_T(\omega) \right)^{- \frac{1}{1 + \tau(\gamma - 1)}}.
    \label{nomeqn:M}
  \end{equation}
  The central bank of the stabilizing country stabilizes its real and
  nominal exchange rate by setting monetary policy according to:
  \begin{equation}
    M^m_2(\omega) = (1 - \tau) \tau^{-2 + \frac{1}{1 + \tau (\gamma - 1)}}
    Y^m_N(\omega)^{-1 + 2 \tau + \gamma \left( 2 (1 - \tau) - \frac{1}{1 + \tau (\gamma - 1)} \right)}
    \left( Z(\omega)\Lambda_T(\omega) \right)^{- \frac{1}{1 + \tau(\gamma - 1)}},
    \label{nomeqn:Mm}
  \end{equation}
  where $Z(\omega)$ is the state-contigent tax and $\Lambda_T(\omega)$
  is the shadow price of traded consumption from baseline model in
  section \ref{sec:real}. The equilibrium allocation then coincides
  exactly with the equilibrium allocation under stabilized exchange
  rates in the baseline model in section \ref{sec:real}. If the
  central bank in the stabilizing country instead also behaves
  according to (\ref{nomeqn:M}), the equilibrium allocation coincides
  with the equilibrum under freely floating exchange rates in section
  \ref{sec:real}.\end{prop}

\begin{proof}
  Plugging \eqref{nomeqn:Mm} into \eqref{nomeqn:PT} and solving for
  $Z(\omega) \Lambda_T(\omega)$ yields:
  \begin{equation*}
    Z(\omega) \Lambda_T(\omega) = \tilde{\Lambda}^m(\omega) \tilde{P}^m_T.
  \end{equation*}
  Moreover, plugging \eqref{nomeqn:M} into \eqref{nomeqn:PT} and
  solving for $\Lambda_T(\omega)$ yields:
  \begin{equation*}
    \Lambda_T(\omega) = \tilde{\Lambda}^n(\omega) \tilde{P}^n_T \text{ for } n = t, o.
  \end{equation*}
  Hence, the first-order conditions \eqref{nomeqn:FOCT} and
  \eqref{nomeqn:FOCN} coincide with the shadow prices of traded and
  nontraded consumption from Appendix \ref{Appendix_ModelDetails}. As
  a result, the allocation of goods must also be the same.
\end{proof}

Equation \eqref{nomeqn:Mm} shows that when the target country
appreciates (i.e. states of the world in which
$Z(\omega)\Lambda_T(\omega)$ is high), the stabilizing country's
central bank contracts its money supply to match the appreciation.
Another way to show this policy is to take the log difference in the
stabilizing country's monetary policy between the stabilizing regime,
$M^m_2(\omega)$ and the freely floating regime (when we replace
$Z(\omega)\Lambda_T(\omega)$ with $\Lambda_T^\ast(\omega)$):
\begin{equation*}
  m^m_2 - m^{m \ast}_2
  = - \frac{1}{1 + \tau(\gamma - 1)} (z + \lambda_T - \lambda_T^\ast).
\end{equation*}
We plug in the expressions for $z$ and $\lambda_T - \lambda_T^\ast$
given by Lemma \ref{lemma:CostofPeg} and equation \eqref{eqn:lambdat2}
to show:
\begin{equation*}
  m^m_2 - m^{m \ast}_2
  = - \frac{\zeta (1 - \theta^m)}{\gamma \tau}
  \left( p^{t \ast} - p^{m \ast} \right).
\end{equation*}

Next, we derive an expression for seigniorage and show the value of
seigniorage accruing to the central bank is equal to $-\Delta Res$,
which is given by equation \eqref{eq:kappacost}. The net present value
of seigniorage is:
\begin{equation*}
  \text{seigniorage} = 
  - \left(\frac{\Delta \tilde{M}^n_1 }{\tilde{P}^n_T}\right)
  - \mathbb{E}\left[ \frac{\Lambda_T(\omega)}{\Lambda_{T, 1}} 
    \left(\frac{\Delta \tilde{M}^n(\omega)}{\tilde{P}^n_T}\right)
  \right],
\end{equation*}
where $\Delta \tilde M^n_1$ and $\Delta \tilde M^n(\omega)$ are given
by the budget constraints \eqref{eqn:bc_nom_1} and
\eqref{eqn:bc_nom_2}. Equation \eqref{eqn:bc_nom_2} shows:
\begin{align*}
  \frac{\Delta \tilde{M}^n(\omega)}{\tilde{P}^n_T}
  & = C^n_T(\omega) + P^l_N(\omega) C^n_N(\omega)
    - \sum_l A^n_l P^l_N(\omega) Y^l_N(\omega) -  Y^n_T \\
  & = C^n_T(\omega) - \left( A^n_n - 1 \right) P^l_N(\omega) Y^n_N(\omega)
    - \sum_{l \neq n} A^n_l P^l_N(\omega) Y^l_N(\omega) - Y^n_T.
\end{align*}
The second equality comes from plugging in the equilibrium condition
$C^n_N = Y^n_N$. We multiply this result with the stochastic discount
factor, $\Lambda_T(\omega) / \Lambda_{T, 1}$, and take expectations:
\begin{equation*}
  \mathbb{E}\left[ 
    \frac{\Lambda_T(\omega)}{\Lambda_{T, 1}}
    \left( \frac{\Delta \tilde{M}^n(\omega)}{\tilde{P}^n_T} \right)
  \right] = \mathbb{E}\left[ 
    \frac{\Lambda_T(\omega)}{\Lambda_{T, 1}} C^n_T(\omega) 
  \right] - \left( A^n_n - 1 \right) Q^n_N - \sum_{l \neq n} A^n_l Q^l_N -
  Y^n_T \mathbb{E}\left[ \frac{\Lambda_T(\omega)}{\Lambda_{T, 1}} \right].
\end{equation*}
Equation \eqref{eqn:bc_nom_1} shows:
\begin{align*}
  \frac{\Delta \tilde{M}^n_1}{\tilde{P}^n_T}
  & = \sum_l A^n_l Q^l_N + Q_K - Q^n_N - Q_K - \kappa^n   \\
  & = \left( A^n_n - 1 \right) Q^n_N + \sum_{l \neq n} A^n_l Q^l_N - \kappa^n. 
\end{align*}
Combining the two previous expressions yields:
\begin{equation*}
  \left(\frac{\Delta \tilde{M}^n_1}{\tilde{P}^n_T}\right) + 
  \mathbb{E}\left[  
    \frac{\Lambda_T(\omega)}{\Lambda_{T, 1}}
    \left( \frac{\Delta \tilde{M}^n(\omega)}{\tilde{P}^n_T} \right)
  \right] =
  \mathbb{E}\left[ \frac{\Lambda_T(\omega)}{\Lambda_{T, 1}} C^n_T(\omega) \right]
  -  Y^n_T \mathbb{E}\left[\frac{\Lambda_T(\omega)}{\Lambda_{T, 1}} \right]
  - \kappa^n.   
\end{equation*}
We plug in the definition of $\kappa^m$ given by equation
\eqref{eqn:defn_kappa} to show:
\begin{align*}
  \text{seigniorage}
  & = - \left(\frac{\Delta \tilde{M}^n_1}{\tilde{P}^n_T}\right) - 
    \mathbb{E}\left[  
    \frac{\Lambda_T(\omega)}{\Lambda_{T, 1}}
    \left( \frac{\Delta \tilde{M}^n(\omega)}{\tilde{P}^n_T} \right)
    \right] \\
  & = \mathbb{E}\left[ \frac{\Lambda^{\ast}_T(\omega)}{\Lambda^{\ast}_{T, 1}}
    C^{n \ast}_T(\omega) \right]
    - \mathbb{E}\left[ \frac{\Lambda_T(\omega)}{\Lambda_{T, 1}} 
    C^n_T(\omega) \right] 
  \\
  & = - \Delta Res.
\end{align*}

\subsection{Extension: Sticky Prices of the Final Consumption Good
  \label{Appendix_RigidCPI}}

This appendix analyzes an extension of the baseline model in section
\ref{sec:real} in which the price of the final consumption bundle is
sticky, rather than the price of the traded good. Similar to appendix
\ref{Appendix_RigidPrices}, central bank policies can exactly
replicate the equilibrium allocation in the baseline model.

We assume there now exists a firm in each country that produces the
final consumption good according to a Cobb-Douglas production
technology:
\begin{equation*}
  Y^n(\omega)
  = \left( C^n_T(\omega) \right)^\tau
  \left( C^n_N(\omega) \right)^{1 - \tau}
\end{equation*}
We assume this price of the final consumption good is sticky,
$\tilde{P}^n(\omega) = 1$. Hence, the final good firm takes the demand
for the final consumption good and solves the following
cost-minimization problem:
\begin{equation*}
  \min \tilde P^n_T(\omega) C^n_T(\omega) + \tilde P^n_N(\omega) C^n_N(\omega)
  \text{ s.t. }
  \left( C^n_T(\omega) \right)^{\tau} \left( C^n_N(\omega) \right)^{1 - \tau} =
  Y^n(\omega).
\end{equation*}
The solution of this cost minimization problem determines the quantity
of traded and non-traded goods the firm purchases to produce final
consumption goods:
\begin{equation}
  C^n_T(\omega) = \left( \frac{\tau}{1 - \tau}
    \frac{\tilde{P}^n_N(\omega)}{\tilde{P}^n_T(\omega)}
  \right)^{1- \tau} Y^n(\omega), \quad
  C^n_N = \left( \frac{1 - \tau}{\tau}
    \frac{\tilde{P}^n_T(\omega)}{\tilde{P}^n_N(\omega)}
  \right)^{\tau} Y^n(\omega),
  \label{nomeqn:firm_soln}
\end{equation}
which also relates the price of the traded and nontraded goods to the
price of the final consumption good:
\begin{equation*}
  \tilde{P}^n(\omega)
  = \tilde{P}^n_T(\omega) C^n_T(\omega) + \tilde{P}^n_N(\omega) C^n_N(\omega)
  = \frac{\left(  \tilde{P}^n_T(\omega) \right)^{\tau}
    \left(  \tilde{P}^n_N(\omega) \right)^{1 - \tau}}{\tau^\tau
    (1 - \tau)^{1 - \tau}}.
\end{equation*}
As a result, we can write the nominal price of the traded good as a
function of the nominal price of the nontraded good:
\begin{equation}
  \tilde{P}^n_{T}(\omega)
  = \tau (1 - \tau)^{\frac{1 - \tau}{\tau}}
  \tilde{P}^n_N(\omega)^{ - \frac{1 - \tau}{\tau}}
  \label{cpieqn:PT}
\end{equation}

Households continue to exhibit utility \eqref{eqn:utility} over
consumption of the final consumption good, $C^n(\omega)$. Just as in
Appendix \ref{Appendix_RigidPrices}, households enter each period with
a fixed quantity of the domestic currency, and all goods consumed in a
given country must be purchased using the domestic currency. In the
first period, households use their currency to purchase stocks. They
continue to face the budget constraint given by equation
\eqref{eqn:bc_nom_1}. In the second period, households face the
cash-in-advance constraint:
\begin{equation}
  \tilde{P}^n C^n(\omega) \le \tilde M^n(\omega)
  \label{eqn:bc_nom_cpi_2}, 
\end{equation}
where
$\tilde M^n(\omega) = \tilde{P}^n_T \left( \sum_l A^n_l
  P^l_N(\omega)Y^l_N(\omega) + Y^n_T \right) + \Delta \tilde
M^n(\omega)$ is the total quantity of domestic currency each country
$n$ household can use in the second period to purchase consumption
goods.

Since the price of the final consumption good is fixed, equation
\eqref{eqn:bc_nom_cpi_2} and market clearing of the final consumption
good implies:
\begin{equation}
  Y^n(\omega) = \tilde M^n(\omega).
  \label{eqn:mktclr_C}
\end{equation}
Hence, equations \eqref{cpieqn:CT}, \eqref{nomeqn:firm_soln},
\eqref{eqn:mktclr_C} and market clearing in nontraded goods allow us
to write $\tilde{P}^n_N(\omega)$ as a function of monetary policy:
\begin{equation}
  \tilde{P}^n_N(\omega) = (1 - \tau)
  \tilde{M}^n(\omega) / Y^n_N(\omega).
  \label{nomeqn:PN_alt}
\end{equation}
Plugging \eqref{nomeqn:PN_alt} into the expression for $C^n_T(\omega)$
in \eqref{nomeqn:firm_soln} shows:
\begin{equation}
  C^n_T(\omega)
  = \left( Y^n_N(\omega) \right)^{- \frac{1 - \tau}{\tau}}
  \tilde{M}^n(\omega)^{\frac{1}{\tau}}.
  \label{cpieqn:CT}
\end{equation}
Equation \eqref{cpieqn:CT} shows changes in the money supply directly
determine the purchases of traded goods in each of the countries.
Hence, the central banks in each of the countries can engage in
monetary policies that directly replicate the equilibrium in the
baseline model in section \ref{sec:real}.

When the target country appreciates, equation \eqref{cpieqn:CT} shows
that the stabilizing country's central bank would decrease
$\tilde M^n(\omega)$ to decrease domestic consumption of traded goods
and match the appreciation in the target country. Hence, in this
extension where the price of the entire domestic consumption bundle is
sticky, the central bank implements a stabilizion by following the
same rule of contracting the domestic money supply whenever the target
country appreciates.

Moreover, the budget constraints in this appendix are exactly the same
as those in Appendix \ref{Appendix_RigidPrices}. Only the location of
the nominal rigidity has changed. As a result, the derivation of
seigniorage accruing to the central bank is exactly the same as in
Appendix \ref{Appendix_RigidPrices}. Hence, the present value of the
seigniorage from stabilization is still equal to $-\Delta Res$.


\subsection{Extension: Segmented Markets and Cash-In-Advance
  Constraint\label{Appendix_CIA}}

This appendix analyzes an alternative monetary friction where prices
are flexible and monetary policy affects real allocations because
financial markets are segmented \citep{AlvarezAtkesonKehoe2002}.
Again, the key takeaway from this exercise is that, even with this
alternate type of monetary friction, a simple nominal stabilization
can implement a real stabilization of the type discussed in Section
\ref{sec:real} of the main text. We will show that when the target
country appreciates, the stabilizing country's central bank matches
the appreciation by contracting the domestic money supply. Finally, we
show the seigniorage from stabilization remains equal to $-\Delta Res$
under this alternative monetary friction.

Each country has a central bank that issues a national currency. All
goods must be paid for in the domestic currency of the country from
which they originate. All households face a cash-in-advance
constraint, and all prices are flexible. Within each country, only a
fraction $\phi$ of households can trade in the international stock
market, label these households `active.' The remaining $1- \phi$ of
households do not have access to financial markets. The central banks
in the target and outside countries use their control of the money
supply to recover the efficient allocation of resources, taking as
given the actions of the stabilizing country's central bank. By
contrast, the central bank in the stabilizing country uses its control
of monetary policy to stabilize the nominal exchange rate.

In the second period, the cash-in-advance constraint for active
households is:
\begin{equation}
  \tilde{P}^n_T(\omega) C^n_T(\omega) + \tilde{P}^n_N(\omega) C^n_N(\omega) 
  \le \tilde{M}^n_1 + \tilde{P}^n_T(\omega) 
  \left( \sum_l A_l^n P_N^l(\omega) Y_N^l(\omega) + Y_T^n \right).
  \label{eqn:seg_bc}
\end{equation}
where $\tilde{P}_T^n$ is the nominal price of the traded good in
country $n$ and $\tilde{M}^n_1$ is the nominal money holding of the
active household in terms of the national currency of its home country
$n$ that is carried over from the first period. Since inactive
households do not have access to financial markets, their cash in
advance constraint in period 2 is:
\begin{equation*}
  \tilde{P}^n_T(\omega) \hat{C}^n_T(\omega) + \tilde{P}^n_N(\omega) \hat{C}^n_N(\omega)
  \le \hat{M}^n_1,
\end{equation*}
where $\hat{M}^n_1$ is the cash holding of an inactive households
carried over from the first period. All households within a given
country start the first period with identical cash holdings,
$\tilde{M}^n_0$. The first-period constraint for active households is
\begin{equation}
  \tilde{M}^n_1 + \tilde{P}^n_{T, 1} 
  \left( \sum_l A^n_l Q^l_N + Q_K K^n_N \right) \le
  \tilde{P}^n_{T, 1} \left( Q^n_N + Q_K + \kappa^n \right) + \tilde{M}^n_0,
  \label{eqn:seg_bc1}
\end{equation}
and the first-period constraint for inactive households is
\begin{equation*}
  \hat{M}^n_1
  \le \tilde{P}^n_{T, 1} \left( Q^n_N + Q_K + \hat{\kappa}^n \right)
  + \hat{M}^n_0.
\end{equation*}

The assumption that all goods must be paid for in the domestic
currency from which they originate implies the money market clearing
condition:
\begin{equation}
  \tilde{P}^n_T(\omega) Y^n_T + \tilde{P}^n_N(\omega) Y^n_N = \bar{M}^n(\omega)
  \label{eqn:rc_money}
\end{equation}
where $\bar{M}^n = \phi \tilde{M}^n + (1 - \phi) \hat{M}^n$ is the
aggregate money supply in country $n$. The central bank changes the
monetary base in the second period through open market operations in
the stock market:
\begin{equation}
  \phi \tilde{P}^n_T(\omega) 
  \left( \sum_l A_l^n P_N^l(\omega) Y_N^l(\omega) + Y_T^n \right) 
  = \bar{M}^n_2 - \bar{M}^n_1 
  = \phi \left( \tilde{M}^n_2 - \tilde{M}^n_1 \right). 
  \label{eqn:seg_policy} 
\end{equation}

Inactive households split their supply of money between traded and
nontraded goods. Their consumption is:
\begin{equation*}
  \hat{C}^n_T (\omega) = \tau \frac{\hat{M}^n_1}{\tilde{P}^n_T(\omega)}
  \text{, and }
  \hat{C}^n_N (\omega) = (1 - \tau) \frac{\hat{M}^n_1}{\tilde{P}^n_N(\omega)}.
\end{equation*}
Because prices are flexible, changes in the money supply affect the
equilibrium allocation in this economy only because it affects the
real purchasing power of these inactive households. That is, the
central bank can affect the allocation by increasing or decreasing the
purchasing power of these households. Define the shock to the real
purchasing power of inactive households in country $n$, controlled by
country \(n\)'s central bank as \begin{equation} \exp(-\mu^n) =
  \frac{1}{P^n(\omega)} \frac{\hat{M}^n_1}{\tilde{P}^n_T(\omega)},
  \label{eqn:def_mu}
\end{equation}
so that a high \(\mu\) corresponds to an expansionary monetary policy,
higher inflation, and lower purchasing power of inactive households.


Given this definition, the consumption of inactive households can be
re-written as:
\begin{equation*}
  \hat{C}^n_T (\omega)
  = \tau \exp(-\mu^n) P^n(\omega)
  \text{, and }
  \hat{C}^n_N (\omega)
  = (1 - \tau) \exp(-\mu^n) P^n(\omega)
  \frac{\tilde{P}^n_T(\omega)}{\tilde{P}^n_N(\omega)}
\end{equation*}


\iffalse By combining equation \eqref{eqn:def_mu} with equation
\eqref{eqn:seg_bc1}, we can express the country-level budget
constraint for country $n$ as:
\begin{equation*}
  \int \frac{\Lambda_T(\omega)}{\Lambda_{T, 1}}
  \left( P^n(\omega)C^n(\omega) + \frac{1 - \phi}{\phi} \exp(-\mu) P^n(\omega) \right)
  g(\omega) d\omega 
  = \frac{1}{\phi} \left( Q_N + Q_k + \kappa^n  + \frac{\tilde{M}^n_0}{\tilde{P}^n_{T,1}}\right).
\end{equation*}
\fi

Active households maximize their expected utility subject to their
budget constraints \eqref{eqn:seg_bc} and \eqref{eqn:seg_bc1}, as well
as the consumption of inactive households. We derive first-order
conditions and log-linearize around the deterministic equilibrium. The
real exchange rate between the stabilizing country and target country
is:
\begin{equation*}
  s^{p, t}
  = \frac{\gamma (1 - \tau)}{\gamma \tau + \phi (1 - \tau)}
  \left( y^t_N - y^p_N \right)
  + \frac{\gamma (1 - \tau)(1 - \phi) }{\gamma \tau + \phi (1 - \tau)}
  \left( \mu^t - \mu^p \right)
\end{equation*}
A positive \(\mu^n\) (high inflation) shifts resources to the active
households in country \(n\) and depreciates the stabilizing country's
real exchange rate.

The real exchange rate under the freely floating regime is:
\begin{equation*}
  s^{m, t \ast} 
  = \frac{\gamma (1 - \tau)}{\gamma \tau + \phi (1 - \tau)}
  \left( y^t_N - y^m_N \right)
\end{equation*}
and the variance of this exchange rate is:
\begin{equation*}
  \text{var}\left[ s^{m, t \ast} \right]
  = \frac{2 \gamma^2 (1 - \tau)^2}{\left( \gamma \tau + \phi (1 - \tau) \right)^2}
  \sigma_N^2
\end{equation*}
The stabilizing country imposes a real exchange rate stabilization of
strength $\zeta$ by choosing:
\begin{align}
  \mu^m
  & = \zeta \frac{1}{1 - \phi} \left( y^t_N - y^m_N \right) \notag \\
  & = \zeta \frac{\gamma \tau + \phi (1 - \tau)}{\gamma (1 - \tau)(1 - \phi)}
    \left( p^{m \ast}  - p^{t \ast} \right) 
\end{align}

The previous equation shows that when the target country appreciates,
the stabilizing country lowers its own inflation rate (i.e. contracts)
to match the appreciation in the target country. Under this
alternative form of monetary friction, lower inflation shifts
resources from the active household towards the inactive household,
which increases the marginal utility of active households and thus
appreciates the real price level in the stabilizing country.

We show this result formally by explicitly solving for the monetary
policy that enforces a nominal exchange rate stabilization. The
nominal exchange rate in this economy is equal to the real exchange
rate plus inflation:
\begin{equation*}
  \tilde{s}^{m, t} = p^m + \mu^m - p^t - \mu^t
\end{equation*}
$\mu^t =\mu^o= 0$ by assumption. However, the nominal exchange rate is
affected by monetary policy through $\mu^m$. We solve for the monetary
policy that implements a nominal exchange rate stabilization of
strength $\tilde \zeta$:
\begin{align}
  \mu^m
  & = \frac{\tilde \zeta \gamma (1 - \tau)}{\gamma (1 - \phi)(1 - \tau)
    - (\gamma \tau + (1 - \tau) \phi)}
    \left( y^t_N - y^m_N \right). \\
  & =\frac{\tilde \zeta \left( \gamma \tau + \phi (1 - \tau) \right)}
    {\gamma (1 - \phi)(1 - \tau) - (\gamma \tau + (1 - \tau) \phi)}
    \left( p^{m \ast} - p^{t \ast} \right)
\end{align}
Under this policy, the stabilizing country's real exchange rate is:
\begin{align*}
  s^{m, t}
  & = \left(
    1 - \frac{\tilde{\zeta} \gamma (1 - \tau)(1 - \phi)}
    {\gamma (1 - \phi)(1 - \tau) - (\gamma \tau + (1 - \tau) \phi)}
    \right)
    s^{m, t \ast} 
\end{align*}
Hence, a policy that implements a nominal stabilization of strength of
$\tilde{\zeta}$ will implement a real stabilization of strength:
\begin{equation*}
  \zeta =\tilde{\zeta} \frac{\gamma (1 - \tau)(1 - \phi)}
  {\gamma (1 - \phi)(1 - \tau) - (\gamma \tau + (1 - \tau) \phi)} 
\end{equation*}
If $\gamma (1 - \phi)(1 - \tau) > (\gamma \tau + (1 - \tau) \phi)$,
then a nominal stabilization implements a stronger real stabilization.

Seigniorage is a function of the present discounted value of the
change in the money supply in both periods:
\begin{equation*}
  \text{seigniorage}
  = - \frac{\bar{M}^n_1 - \bar{M}^n_0}{\tilde{P}^n_{T, 1}}
  - \mathbb{E}\left[ \frac{\Lambda_T(\omega)}{\Lambda_{T, 1}}
    \left( \frac{\bar{M}^n_2(\omega) - \bar{M}^n_1}{\tilde{P}^n_T(\omega)} \right)\right].
\end{equation*}
Following the same calculations as in Appendix
\ref{Appendix_RigidPrices}, we can show:
\begin{align*}
  \text{seigniorage}
  & = \mathbb{E}\left[
    \left( \frac{\Lambda^{\ast}_T(\omega)}{\Lambda^{\ast}_{T, 1}} \right)
    \left( \phi C^{m \ast}_T(\omega) + (1 - \phi) \hat{C}^{m \ast}_T(\omega) \right)
    \right] \\
  & \quad \quad - \mathbb{E}\left[
    \left( \frac{\Lambda_T(\omega)}{\Lambda_{T, 1}} \right)
    \left( \phi C^m_T(\omega) + (1 - \phi) \hat{C}^m_T(\omega) \right)
    \right]
\end{align*}
where asterisks denote an equilibrium in which the stabilizing country
does not actively manipulate the variance of the exchange rate. In the
segmented markets model, seigniorage is still equal to change in the
value of traded consumption ($-\Delta Res$). However, seigniorage in
the segmented markets model takes into account the consumption of both
active and inactive households.


\section{Model Extensions}

\subsection{Partial exchange rate stabilization
  \label{app:partstab}}

This appendix formalizes the effects of partial exchange rate
stabilization. In a first step, we use the partition defined in the main text to write the
variance of exchange rates in the freely floating regime as
\begin{equation}
  \begin{split}
    \text{var}[s^{\ast m,t}] &= \int_{\Omega} \left(s^{\ast m,t}-
      \mathbb{E}[s^{\ast m,t}|\{K_n\}]\right)^2g(\omega) d\omega
    \\
    &= \int_{\Omega_s} \left(s^{\ast m,t}-\mathbb{E}[s^{\ast m,t}|\{K_n\}]\right)^2 g(\omega)d\omega+\int_{\Omega_{-s}} \left(s^{\ast m,t}-\mathbb{E}[s^{\ast m,t}|\{K_n\}]\right)^2g(\omega)d\omega\\
    &=\text{Prob}\left[\omega\in\Omega_s\right]
    \text{var}\left[s^{\ast
        m,t}|\Omega_s\right]+\text{Prob}\left[\omega\in\Omega_{-s}\right]
    \text{var}\left[s^{\ast m,t}|\Omega_{-s}\right]
  \end{split}
\end{equation}
since the conditional means in the two subregions of the state space
are identical. By the same token, partial stabilization delivers a
variance of the exchange rate of
\begin{equation}
  \begin{split}
    \text{var}[s^{m,t}]&= \text{Prob}\left[\omega\in\Omega_s\right] \text{var}\left[s^{m,t}|\Omega_s\right]+\text{Prob}\left[\omega\in\Omega_{-s}\right] \text{var}\left[s^{m,t}|\Omega_{-s}\right]\\
    &=\text{Prob}\left[\omega\in\Omega_s\right] \text{var}\left[(1-\zeta)(s^{\ast m,t}-\mathbb{E}[s^{\ast m,t}|\{K_n\}])|\Omega_s\right]+\text{Prob}\left[\omega\in\Omega_{-s}\right] \text{var}\left[s^{\ast m,t}|\Omega_{-s}\right]\\
    &=\text{Prob}\left[\omega\in\Omega_s\right] (1-\zeta)^2\text{var}\left[s^{\ast m,t}|\Omega_s\right]+\text{Prob}\left[\omega\in\Omega_{-s}\right] \text{var}\left[s^{\ast m,t}|\Omega_{-s}\right]\\
    &<\text{var}\left[s^{\ast m,t}\right].
  \end{split}
\end{equation}

With exchange rate stabilization of strength $\zeta$, the interest
rate differential given by equation (\ref{eq_UIP_RF}) becomes
\begin{equation*}
  \begin{split}
    r^m+ \Delta \mathbb{E}[s^{m,t}]-r^t&=-\text{cov}\left[\lambda_T,s^{m,t}\right]\\
    &=-\text{cov}\left[\lambda_T,(1-\zeta)s^{\ast m,t}\right]\\
    &=-(1-\zeta)\text{cov}\left[\lambda_T,s^{\ast m,t}\right].
  \end{split}
\end{equation*}
The effects of partial stabilization for interest rate differentials
work in the same direction. Again using the fact that the conditional
means are identical in the two subregions, we decompose the covariance into the following
terms:
\begin{equation*}
  \small
  \begin{split}
    r^m+ \Delta
    \mathbb{E}[s^{m,t}]-r^t&=-\text{cov}\left[\lambda_T,s^{m,t}\right]
    = -\int_{\Omega} \left(\lambda_T - \mathbb{E}\left[\lambda_T|\{K_n\}\right]\right) \left(s^{m,t}-\mathbb{E}\left[s^{m,t}|\{K_n\}\right]\right)g(\omega)d\omega\\
    &= -\text{Prob}\left[\omega\in\Omega_s\right] \int_{\Omega_s} \left(\lambda_T - \mathbb{E}\left[\lambda_T|\{K_n\}\right]\right) \left(s^{m,t}-\mathbb{E}[s^{m,t}|\{K_n\}]\right)g_s(\omega)d\omega\\
    &\quad -\text{Prob}\left[\omega\in\Omega_{-s}\right] \int_{\Omega_{-s}}\left(\lambda_T - \mathbb{E}\left[\lambda_T|\{K_n\}\right]\right) \left(s^{m,t}-\mathbb{E}[s^{m,t}|\{K_n\}]\right)g_{-s}(\omega)d\omega\\
    &= -\text{Prob}\left[\omega\in\Omega_s\right] \int_{\Omega_s} \left(\lambda_T - \mathbb{E}\left[\lambda_T|\Omega_s,\{K_n\}\right]\right) \left(s^{m,t}-\mathbb{E}[s^{m,t}|\{K_n\}]\right)g_s(\omega)d\omega\\
    &\quad -\text{Prob}\left[\omega\in\Omega_s\right] \left(
      \mathbb{E}\left[\lambda_T|\Omega_s,\{K_n\}\right]-\mathbb{E}\left[\lambda_T|
        \{K_n\}\right]\right) \int_{\Omega_s} \left(s^{m,t}-\mathbb{E}[s^{m,t}|\{K_n\}]\right)g_s(\omega)d\omega\\
    &\quad -\text{Prob}\left[\omega\in\Omega_{-s}\right] \int_{\Omega_{-s}} \left(\lambda_T - \mathbb{E}\left[\lambda_T|\Omega_{-s},\{K_n\}\right]\right) \left(s^{m,t}-\mathbb{E}[s^{m,t}|\{K_n\}]\right)g_{-s}(\omega)d\omega\\
    &\quad -\text{Prob}\left[\omega\in\Omega_{-s}\right] \left(
      \mathbb{E}\left[\lambda_T|\Omega_{-s},\{K_n\}\right]-\mathbb{E}\left[\lambda_T|
        \{K_n\}\right]\right) \int_{\Omega_{-s}} \left(s^{m,t}-\mathbb{E}[s^{m,t}|\{K_n\}]\right)g_{-s}(\omega)d\omega\\
    &= -\text{Prob}\left[\omega\in\Omega_s\right] \int_{\Omega_s} \left(\lambda_T - \mathbb{E}\left[\lambda_T|\Omega_s,\{K_n\}\right]\right) \left(s^{m,t}-\mathbb{E}[s^{m,t}|\{K_n\}]\right)g_s(\omega)d\omega\\
    &\quad -\text{Prob}\left[\omega\in\Omega_{-s}\right] \int_{\Omega_{-s}} \left(\lambda_T - \mathbb{E}\left[\lambda_T|\Omega_{-s},\{K_n\}\right]\right) \left(s^{m,t}-\mathbb{E}[s^{m,t}|\{K_n\}]\right)g_{-s}(\omega)d\omega\\
    &= -\text{Prob}\left[\omega\in\Omega_s\right]
    \text{cov}\left[\lambda_T,s^{m,t}|\Omega_s\right]-\text{Prob}\left[\omega\in\Omega_{-s}\right]
    \text{cov}\left[\lambda_T,s^{m,t}|\Omega_{-s}\right],
  \end{split}
\end{equation*}
where $g_s(\omega) = \dfrac{g(\omega)}{\text{Prob}\left[\omega\in\Omega_s\right]}$ and $g_{-s}(\omega) = \dfrac{g(\omega)}{\text{Prob}\left[\omega\in\Omega_{-s}\right]}$. The second-to-last step follows from the fact that the conditional means are identical and thus $\mathbb{E}\left[s^{m,t}-\mathbb{E}\left[s^{m,t}|\{K_n\}\right]|\Omega_s \right] = 0$.\\
With partial exchange rate stabilization, we get
\begin{equation*}
  \begin{split}
    r^m+ \Delta \mathbb{E}[s^{m,t}]-r^t&=-\text{cov}\left[\lambda_T,s^{m,t}\right]\\
    &= -\text{Prob}\left[\omega\in\Omega_s\right] \text{cov}\left[\lambda_T,s^{m,t}|\Omega_s\right]-\text{Prob}\left[\omega\in\Omega_{-s}\right] \text{cov}\left[\lambda_T,s^{m,t}|\Omega_{-s}\right]\\
    &= -\text{Prob}\left[\omega\in\Omega_s\right]
    (1-\zeta)\text{cov}\left[\lambda_T,s^{\ast m,t}|\Omega_s\right]
    -\text{Prob}\left[\omega\in\Omega_{-s}\right]
    \text{cov}\left[\lambda_T,s^{\ast m,t}|\Omega_{-s}\right].
  \end{split}
\end{equation*}
Rearranging the last equation to
\begin{equation*}
  r^m+\Delta \mathbb{E}[s^{m,t}]-r^t=-\text{cov}\left[\lambda_T,s^{m,t}\right]=  -\text{cov}\left[\lambda_T,s^{\ast m,t}\right] +\zeta \text{Prob}\left[\omega\in\Omega_s\right]\text{cov}\left[\lambda_T,s^{\ast m,t}|\Omega_s\right],
\end{equation*}
we see that the effects of partial stabilization are a milder version
of currency stabilization discussed previously. In fact, partial
stabilization of strength $\zeta$ in a subset of the state space
corresponds to currency stabilization of strength
$\zeta
\text{Prob}\left[\omega\in\Omega_s\right]\text{cov}\left[\lambda_T,s^{\ast
    m,t}|\Omega_s\right]$.

\subsection{Stabilization Relative to a Basket of
  Currencies\label{Appendix_baskets}}

Our analysis above also extends directly to stabilizations relative to
a basket of currencies. Consider a country that wishes to stabilize
its real exchange rate with the basket
\begin{equation*}
  p^b = (1 - w) p^t + w p^o
\end{equation*}
where $w$ is the basket's weight on the outside country and $1 - w$
the weight on the target country. Using (\ref{eq_UIP_RF}), it is then
easy to show that stabilizing relative to a basket of currencies has
effects akin to a stabilization relative to a (hypothetical) country
with a weighted average size of the basket's constituents:
\begin{equation*}
  r^m + \Delta\mathbb{E} s^{m,o} - r^o =  
  \left(r^{t \ast} + \Delta\mathbb{E}s^{t, o \ast} - r^{o \ast} \right) - \zeta \frac{\gamma (1 - \tau)^2 \left( (\bar{\theta} - \theta^m) (\gamma - 1) \tau + \theta^m (2 - w - 2 \bar{w} \zeta) \right)}{\tau \left( 1 + (\gamma - 1) \tau \right)} \sigma_N^2
\end{equation*}
where $\bar{\theta} = w \theta^t + (1 - w) \theta^o$ is the weighted
average size of the basket's constituents and
$\bar{w} = 1 - (1 - w) w$ is a positive constant less than one.

Although clearly a less effective means of lowering domestic interest
rates than stabilizations relative to the largest economy in the
world, stabilizing relative to a basket may be appealing for some
countries, because it reduces price impact. When stabilizing relative
to a basket, the stabilizing country's exports are less sensitive to
shocks affecting only one of the two other countries, decreasing the
volatility of its exports and thus lowering the stabilization's impact
on world-market prices. For a large country, stabilizing relative to a
basket may thus be cheaper to implement than stabilizing relative to
the largest economy in the world.


\subsection{Feedback between Risk Premia and Capital
  Accumulation \label{Appendix_EndogenousCapital}}


In this appendix, we show Propositions \ref{prop:KNRBCPeg} through
\ref{prop:KNRBCTarget} continue to hold when we solve explicitly for
the feedback between risk premia and capital accumulation. First, note
that changes in the level of capital accumulation affect the expected
level of consumption, but not the conditional covariance of
consumption across countries in our log-linear solution. It follows
immediately that all statements in Propositions \ref{prop:KNRBCPeg}
through \ref{prop:KNRBCTarget} that depend on the covariances between
asset payoffs and the shadow price of traded goods are unchanged. That
is, all statements regarding interest differentials, expected currency
returns, and the world-market value of domestic firms continue to
hold.

Second, to show that all statements in Propositions
\ref{prop:KNRBCPeg} through \ref{prop:KNRBCTarget} pertaining to the
capital stock itself continue to hold when we solve explicitly for the
feedback between risk premia and capital accumulation. To this end, we
use the second-order approximation of the Euler equation for capital
accumulation \eqref{eqn:FOCK}:
\begin{equation*}
  \lambda_{T, 1}  + q_K + k^n
  = \log[\nu] + \mathbb{E}\left[ \lambda_N^n + y_N^n \right]
  + \frac{1}{2} \text{var}\left[ \lambda_N^n + y_N^n \right] \quad \forall n,
\end{equation*}
and the log-linear resource constraint for capital:
\begin{equation*}
  0 = \sum_n \theta^n k^n.
\end{equation*}
We plug in the expression for
$\lambda_N^n = p^n_N + q + \lambda^n_{T, 1}$ from Appendix
\ref{Appendix_Loglinear} to write $\lambda_N^n$ as a function of
$y_N^n$, and then we plug in $y_N^n = \eta + \nu k^n$ to write the
$y_N^n$ as a function of the capital stock and the productivity shock.
We solve this system of four equations for $k^m, k^t, k^o$ and $q_K$.

In a freely floating exchange rate economy $(\zeta = 0)$, we find
\begin{equation*}
  k^{m \ast} - k^{t \ast} = 
  \frac{(\gamma - 1)^3 (1 - \tau)^2 \tau}{\left( 1 + (\gamma - 1) \tau \right) \left( 1 + (\gamma - 1) (1 - \tau) \nu + (\gamma - 1) \tau \right)} 
  \left( \theta^m - \theta^t \right) \sigma_N^2. 
\end{equation*}
Comparing this expression with $k^m - k^t$ derived from solving the
system of four equations above shows that allowing for feedback
between risk premia and capital accumulation merely reduces the size
of the difference in capital accumulation by a constant factor smaller
than one, leaving the economic insights of our analysis unaffected.

The same is true for the equivalent expression under stabilized
exchange rates, though this factor is too large to reproduce in print.
For the special case of \(\zeta=1\), we can show
\begin{equation*}
  \left( k^m - k^t \vert_{\zeta = 1} \right) = k^{m \ast} - k^{t \ast} + 
  \frac{(\gamma - 1)^3 (1 - \tau)^2 \tau \left( \theta^t - \theta^m \right) \sigma_N^2}{\left( 1 + (\gamma - 1) \tau \right) \left( 1 + (\gamma - 1) (1 - \tau) \nu + (\gamma - 1) \tau \right)} 
  = 0.
\end{equation*}



\section{Appendix to Section \ref{sec:full_model}}

In this appendix, we provide additional details about the model in
section \ref{sec:full_model} and formally derive its equilibrium
conditions. To avoid solving the optimization problem
separately for households in the stabilizing country and households in
the rest of the world, we generalize the notation to allow all
countries to impose state-contingent taxes, $Z^n(\omega)$, and provide
lump sum transfers, $\bar{Z}^n$. The governments in the target and
outside countries do not use these instruments, such that
$Z^t(\omega)=Z^o(\omega) = 1$ and $\bar{Z}^t =\bar{Z}^o= 0$.

\subsection{Equilibrium Consumption
  \label{Appendix_Inactive} \label{Appendix_Active}}

Inactive households in country $n$ maximize utility, defined in
equation (\ref{eqn:utility_gen}), in each state of the world by
splitting their wealth $\exp(-\mu^n) P^n(\omega)$ optimally between
traded and nontraded consumption,
\begin{align*}
  \max_{\hat{C}_T^n\left( \omega \right), \hat{C}_N^n\left( \omega\right)}
  & \frac{1}{1 - \gamma}  
    \left(\exp\left( \chi^n\right) \left( \hat{C}_T^n \left( \omega \right) \right)^{\tau} 
    \left( \hat{ C }_N^n\left( \omega \right) \right)^{1 - \tau} \right)^{1 - \gamma} \\
  \text{ s.t. }
  & \hat{C}_T^n ( \omega ) + P^n_{N}(\omega) \hat{C}_N^n ( \omega ) \le 
    \exp(-\mu^n)P^n(\omega), \notag
\end{align*}
where hats indicate consumption by inactive households. We solve this
problem by setting up a Lagrangian and taking first-order conditions
with respect to $\hat{C}_T^n( \omega )$ and $\hat{C}_N^n ( \omega )$.
Inactive households then optimally consume the following bundle of
traded and nontraded goods,
\begin{equation*}
  \hat{C}_{T}^n(\omega) = \exp (- \mu^n) \tau P^n(\omega) \text{, }  \quad
  \hat{C}_{N}^n(\omega) = \exp ( - \mu^n) \frac{(1 - \tau)P^n(\omega) }{P_{N}^n(\omega) }.
\end{equation*}

Active households own all the productive assets within the country and
are short the nominal bonds owned by inactive households. They
maximize their utility (\ref{eqn:utility_gen}) subject to their
intertemporal budget constraint:
\begin{align}
  & \mathbb{E}\left[ \frac{\Lambda_T (\omega)}{\Lambda_{T, 1}} \left(
    Z^n(\omega) C^n_T(\omega) + P^n_N(\omega) C^n_N(\omega)
    + \frac{1 - \phi}{\phi} P^n(\omega) e^{- \mu^n}
    \right) \right] \label{eqn:bc_peg_full}  \\
  & \le \frac{1}{\phi} \left( Q_K - Q_K K^n
    + \mathbb{E}\left[ \frac{\Lambda_T (\omega)}{\Lambda_{T, 1}}
    \left( P^n_N(\omega) Y^n_N + Y^n_T \right) \right] +
    \kappa^n  + \bar{Z}^n \right), \notag
\end{align}
where $(1 - \phi) / \phi$ is the number of inactive households per
active household in each country and endowments are adjusted by a
factor $1 / \phi$ because active households now own proportionally
more productive assets per capita; \(\kappa^n \) again denotes the
transfer that decentralizes the allocation corresponding to the social
planner's problem with unit Pareto weights under freely floating
exchange rates. In the stabilizing country, the government use the
lump-sum transfer, $\bar{Z}^m$, to equalize the marginal utility of
wealth between the stabilizing country and the rest of the world
(\ref{eqn:P2}).

The first-order conditions of the active households' problem are:
\begin{align}
  \frac{\tau \exp( (1 - \gamma) \chi^n) \left( C^n \right)^{1 - \gamma} \left( C_T^n \right)^{-1}}{Z^n(\omega)}
  &= \Lambda_{T, 1} Q(\omega) \label{eqn:FOCCT_full_gen}\\ 
  (1 - \tau) \exp( (1 - \gamma) \chi^n) \left( C^n \right)^{1 - \gamma} \left( C_N^n \right)^{-1}
  &= \Lambda_{T, 1} Q(\omega) P_N^n(\omega) \label{eqn:FOCCN_full_gen}.
\end{align}
Analogous to Appendix \eqref{Appendix_Loglinear}, we find it
convenient to denote the stochastic discout factor with
$Q(\omega) = \Lambda_T(\omega) / \Lambda_{T, 1}$. The first-order
condition with respect to capital accumulation is
\begin{equation}
  Q_K  = \mathbb{E} \left[
    \frac{\Lambda_T (\omega)}{\Lambda_{T, 1}}
    P_N^n(\omega) e^{\eta^n} \nu \left( K^n \right)^{\nu - 1} 
  \right].
  \label{eqn:FOCK_full_gen}
\end{equation}

\subsection{Log-linearized System of
  Equations \label{Appendix_Loglinear_FullModel}}

We next derive log-linearized first-order conditions. Equation
(\ref{eqn:RCT}) defines the resource constraint for traded goods.
Equation (\ref{eqn:RCN}) defines the (three) resource constraints for
nontraded goods in each country, and equation (\ref{eqn:RCK}) defines
the resource constraint for capital goods. Equations
(\ref{eqn:FOCCT_full_gen}) and (\ref{eqn:FOCCN_full_gen}) define the
three first-order conditions with respect to traded consumption and
the three first-order conditions with respect to nontraded
consumption. Equation (\ref{eqn:FOCK_full_gen}) defines the three
Euler equations for capital investment in each country. In total, we
derive a system of 14 equations. To study the model in closed form, we
again log-linearize around the deterministic solution --- the point at
which the variances of shocks are zero
$\left( \sigma_{N, n} = 0 \right)$ and all firms have a capital stock
fixed at the deterministic steady-state level. To simplify the
exposition, we thus again ignore the feedback effect of differential
capital accumulation on the size of risk premia, studying the
\textit{incentives} to accumulate different levels of capital across
countries, while holding the capital stock fixed. The log-linear
first-order conditions are:
\begin{align*}
  (1 - \gamma) \chi^n + (1 - \gamma) \left( \tau c_T^n + (1 - \tau) c_N^n \right) - c_T^n + 
  \log \tau &= z^n + q + \lambda_{T, 1} \\
  (1 - \gamma) \chi^n + (1 - \gamma) \left( \tau c_T^n + (1 - \tau) c_N^n \right) - c_N^n + 
  \log(1 -  \tau) &= p^n_N + q + \lambda_{T, 1}.
\end{align*}
Similar to Appendix \ref{Appendix_ModelDetails}, let
$q = \lambda_T - \lambda_{T, 1}$ denote the stochastic discount
factor. Also recall that the transfes
\(\{\kappa^n\},\bar{Z}^m$ equalize the first-period Lagrange
multiplier $\lambda_{T,
  1}$ across active households in all countries. The log-linear
approximation of equation (\ref{eqn:FOCK_full_gen}) is:
\begin{equation*}
  \lambda_{T, 1} + q_K + k^n
  = \log[\nu] + \mathbb{E}\left[ \lambda_N^n + y_N^n \right]
  + \frac{1}{2} \text{var}\left( \lambda_N^n + y_N^n \right).
\end{equation*}
The log-linear resource constraints are:
\begin{align*}
  \phi c_N^n + (1 - \phi) \left( - \mu^n - \tau \left( \lambda_N^n - \lambda_T - \log\left( \frac{1 - \tau}{\tau} \right) \right) \right) & = \eta^n + \nu k^n = y_N^n,\\ 
  \sum_{n = m, t, o} \theta^n \left[ \phi c_T^n + (1 - \phi) \left( - \mu^n - (1 - \tau) \left( \lambda_N^n - \lambda_T - \log\left( \frac{1 - \tau}{\tau} \right) \right) \right) \right] & = \sum_{n = m, t, o} \theta^n y_{T, 1}^n = 1, \\
  \sum_{n = m, t, o} \theta^n k^n &= 1.
\end{align*}
This set of fourteen equations allows us to solve for the following
fourteen unknowns
$\left\{k^n, c_N^n, c_T^n, \lambda_N^n \right\}_{n = m, t,
  o}$, $\lambda_{T, 1}$ and $q$.

\subsection{Cost of Stabilization
  \label{Appendix_ProofCostofPeg_FullModel}}

First, we solve for the state-contingent taxes that implement the real
exchange rate stabilization in the model in section
\ref{sec:full_model}, and then we derive an expression for the cost of
the peg. Throughout, we can recover the results in Appendix
\ref{Appendix_ProofCostofPeg} by removing the market segmentation
($\phi = 1$), by setting the monetary shocks to zero ($\mu^n = 0$) and
by setting the preference shocks to zero ($\chi^n = 0$).

Analogous to Appendix \ref{Appendix_ProofCostofPeg}, we search for a
state-contingent tax of the form
\begin{equation*}
  Z(\omega) = \left( \frac{Y_N^m}{Y_N^t} \right)^{a_1}
  \left( \frac{\exp \left( -\mu^t \right)}{\exp \left( -\mu^m \right)}\right)^{a_2} 
  \left( \frac{\exp \left( \chi^m \right)}{\exp \left( \chi^t \right)} \right)^{a_3}.
\end{equation*}
In logs, this state-contingent tax is
\begin{equation*}
  z = a_1 \left( y_N^t - y_N^m \right) + a_2 \left( -\mu^t + \mu^m \right) +
  a_3 \left( \chi^t - \chi^m \right).
\end{equation*}
We follow the procedure from Appendix \ref{Appendix_ProofCostofPeg} to
derive the coefficients $a_1, a_2$ and $a_3$ that stabilize the
exchange rate. The following lemma summarizes these results.
\begin{lemma}
  In the model in section \ref{sec:full_model}, where real exchange
  rates fluctuate in response to monetary shocks, preference shocks,
  and productivity shocks, a tax on the consumption of traded goods in
  the stabilizing country of the form
  \begin{equation*}
    z(\omega) = 
    \frac{\zeta (1 - \tau)}{\tau \left( \tau + \phi (1 - \tau) \right)}
    \left( y_N^m - y_N^t\right) + 
    \frac{(1 - \tau)(1 - \phi)}{\tau \left( \tau + \phi (1 - \tau) \right)}
    \left( \mu^m - \mu^t \right) + 
    \frac{(\gamma - 1)(1 - \tau) \phi}{\gamma \tau \left( \tau + \phi (1 - \tau) \right)}
    \left( \chi^m - \chi^t \right)
  \end{equation*}
  implements a real exchange rate stabilization of strength $\zeta$.
\end{lemma}

Next, we derive the cost of the stabilization. We start with the
budget constraint of the active household in the stabilizing country
given by equation (\ref{eqn:bc_peg_full}), and we identify the
components of the lump-sum transfer, $\bar{Z}$. Following the same
calculations as in Appendix \ref{Appendix_ProofCostofPeg}, we show:
\begin{equation}
  \begin{split}
    \Delta Res & = \mathbb{E}\left[
      \frac{\Lambda_T(\omega)}{\Lambda_{T, 1}} \left( \phi
        C_T^m(\omega) + (1 - \phi) \hat{C}_T^m(\omega) \right)
    \right] \\
    & \quad \quad - \mathbb{E}\left[
      \frac{\Lambda_T(\omega)}{\Lambda_{T, 1}} \left( \phi C_T^{m
          \ast}(\omega) + (1 - \phi) \hat{C}_T^{m \ast}(\omega)
      \right) \right]
  \end{split}
  \label{eqn:delta_res_full}
\end{equation}
In the model in section \ref{sec:full_model}, the cost of
stabilization is thus the change in the value of the stabilizing
country's total consumption of traded goods by active and inactive
households.


\subsection{Proof of Proposition \ref{prop:FullModelResults}
  \label{Appendix_FullModelProof}}

We first prove results for the internal effects of a real exchange
rate stabilization. We plug the log-linear expressions derived from
solving the system of equations in Appendix
\ref{Appendix_Loglinear_FullModel} into equation (\ref{eq_UIP_RF}). We
can then write the interest rate differential between the stabilizing
country and the target as
\begin{align*}
  r^m + \Delta \mathbb{E}s^{m, t} - r^t =
  & \left(r^{m \ast} + \Delta \mathbb{E}s^{m, t \ast} - r^{t \ast} \right) -
    \zeta \frac{\gamma (1 - \tau)^2\left( (\theta^t - \theta^m) \tau (\gamma - \phi) + 2 \phi \theta^m (1 - \zeta) \right)}{\tau \phi \left( \gamma \tau + (1 - \tau) \phi \right)} \sigma_N^2 \\
  & - \zeta \frac{\gamma (1 - \tau)(1 - \phi)^2\left( (\theta^t - \theta^m) \gamma \tau + 2 \phi \theta^m (1 - \zeta) (1 - \tau) \right)}{\tau \phi \left( \gamma \tau + (1 - \tau) \phi \right)} \tilde{\sigma}^2 \\
  & - \zeta \frac{\phi (1 - \tau)(1 - \gamma)^2\left( (\theta^t - \theta^m) \gamma \tau + 2 \phi \theta^m (1 - \zeta) (1 - \tau) \right)}{\tau \gamma \left( \gamma \tau + (1 - \tau) \phi \right)} \sigma_{\chi}^2,
\end{align*}
which implies the exchange rate stabilization decreases the risk-free
rate in the stabilizing country relative to the risk-free rate in the
target country if the target country is larger than the stabilizing
country, $\theta^t > \theta^m$.

Plugging in the log-linear expressions for $p^m_N$, $p^t_N$, and
$\lambda_T$ into equation \eqref{eqn:kspread}, we again find that the
relative incentives to accumulate capital in the stabilizing country
increase with the size of the target country. Because the closed-form
solution equivalent to the one above is too large to print, it is
easier to prove this statement by showing that relative incentives to
accumulate capital increase linearly in \(\theta^t\):
\begin{align*}
  \frac{d}{d \theta^t}
  & \left[ \left( k^m - k^t \right) - \left( k^{m \ast} - k^{t \ast} \right) \right]
    = \frac{\zeta (\gamma - 1) (1 - \tau)^2 \tau (\gamma - \phi)^2}{\left( \phi + (1 - \phi) \tau \right)\left( \gamma \tau + (1 - \tau) \phi \right)}\sigma_N^2 \\
  & + \frac{\zeta (\gamma - 1) \gamma (1 - \tau) \tau (\gamma - \phi) (1 - \phi)^2}{\left( \phi + (1 - \phi) \tau \right)\left( \gamma \tau + (1 - \tau) \phi \right)}\tilde{\sigma}^2 
    + \frac{\zeta (\gamma - 1)^3 (1 - \tau) \tau (\gamma - \phi) \phi^2}{\gamma \left( \phi + (1 - \phi) \tau \right)\left( \gamma \tau + (1 - \tau) \phi \right)} \sigma_{\chi}^2 > 0. 
\end{align*}
It follows immediately that there exists some $\theta_{min}>0$ such
that stabilizing the real exchange rate relative to any country larger
than $\theta_{min}$ will increase the incentives to accumulate capital
in the stabilizing country. Analogous to Appendix
\ref{Appendix_KNRBCPeg}, the spread between the value of the firm in
the stabilizing and target countries yields the same expression as the
right-hand side of equation \eqref{eqn:kspread}. Hence, we have
already shown the value of the firm in the stabilizing country
increases relative to the target country if $\theta^t$ is large
enough.

Because firms are competitive, wages are given by the marginal product
of labor. Hence, an exchange rate stabilization relative to a
sufficiently large target country increases wages in the stabilizing
country relative to all other countries, concluding the proof of the
first statement in Proposition \ref{prop:FullModelResults}.

Next, we derive the cost of stabilization. We calculate changes in the
log value of traded consumption in the stabilizing country given by
\eqref{eqn:delta_res_full}. The log-linear approximation of the total
traded consumption in the stabilizing country from active and inactive
households is: $\phi c_T^{m} + (1 - \phi) \hat{c}_T^m$. We calculate:
\begin{equation*}
  \log \Delta Res = v_T - v^\ast_T,
\end{equation*}
where we use the following second-order approximation of the log value
of total traded consumption:
\begin{equation*}
  v_T = \mathbb{E}\left[ \lambda_T - \psi_T + \phi c_T^{m} + (1 - \phi) \hat{c}_T^m  \right]
  + \frac{1}{2} \text{var}\left[ \lambda_T - \psi_T + \phi c_T^{m} + (1 - \phi) \hat{c}_T^m \right]
\end{equation*}
When the stabilizing country is small ($\theta^m = 0$), the cost of
the stabilization decreases as the target country gets larger:
\begin{align*}
  \frac{d}{d \theta^t}\left( v_T - v_T^{\ast} \right) = -
  & \zeta \frac{(1 - \tau)(1 - \phi)^2 (\gamma - \phi)}{\left( \phi + (\gamma - \phi) \tau\right)^2 }\tilde{\sigma}^2 
    - \zeta \frac{(1 - \tau)^2 (\gamma - \phi)^2}{\left(\phi + (\gamma - \phi) \tau\right)^2} \sigma_N^2 \\  
  & - \zeta \frac{(\gamma - 1)^2 (1 - \tau)(\gamma - \phi) \phi^2}{\gamma \left( \phi + (\gamma - \phi) \tau \right)^2} \sigma_{\chi}^2 < 0.
\end{align*}
Hence, it is cheaper to stabilize relative to a larger country.

Finally, we prove results for the external effects of a real exchange
rate stabilization. Using equation (\ref{eq_UIP_RF}) and the solution
of the model from Appendix \ref{Appendix_Loglinear_FullModel}, we can
write interest rate differential between the target country and the
outside country as
\begin{align*}
  r^t + \Delta \mathbb{E}s^{t, o} - r^o =
  & \left(r^{t \ast} + \Delta \mathbb{E}s^{t, o \ast} - r^{o \ast} \right) + 
    \frac{\zeta \theta^m \gamma (1 - \tau)^2 }{\tau \left( \gamma \tau + \phi (1 - \tau) \right)} \sigma_N^2 
    + \frac{\zeta \theta^m \gamma (1 - \tau)^2 (1 - \phi)^2}{\tau \left( \gamma \tau + \phi (1 - \tau) \right)} \tilde{\sigma}^2 \\
  & + \frac{\theta^m \zeta (\gamma - 1)^2 (1 - \tau)^2 \phi^2}{\gamma \tau \left( \gamma \tau + (1 - \tau) \phi \right)}\sigma_\chi^2,
\end{align*}
which implies the exchange rate stabilization increases the risk-free
rate in the target country relative to the risk-free rate in the
outside country.

We plug the log-linear expressions for $p^t_N$, $p^o_N$ and
$\lambda_T$ into \eqref{eqn:kspread} to derive the differential
incentive to accumulate capital in the target country relative to the
outside countries:
\begin{align*}
  k^t - k^o =
  & k^{t \ast} - k^{o \ast} 
    - \frac{\theta^m \zeta (1 - \tau)^2 (\gamma - \phi)^2}{\left( \gamma \tau + (1 - \tau) \phi \right)^2}
    \sigma_N^2 
    - \frac{\theta^m \gamma \zeta (1 - \tau) (\gamma - \phi) (1 - \phi)^2}{\left( \gamma \tau + (1 - \tau) \phi \right)^2} \tilde{\sigma}^2 - \\
  & \frac{\theta^m (\gamma - 1)^2 \zeta (1 - \tau)(\gamma - \phi) \phi^2}{\gamma \left( 
    \gamma \tau + (1 - \tau) \phi \right)^2}\sigma_{\chi}^2.
\end{align*}
Incentives to accumulate capital in the target country thus decrease
relative to the outside country. Since the marginal product of labor
rises with the level of capital accumulation, the exchange rate
stabilization decreases wages in the target country relative to all
other countries.



