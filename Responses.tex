\setcounter{table}{0} %\captionsetup{tablename= Response Table}
\setcounter{figure}{0} %\captionsetup{figurename= Response Figure}
\setcounter{page}{1}

%\nesponse Figure}
\setcounter{page}{1}

\noindent \bigskip

\bigskip

\bigskip

\noindent \today

\bigskip

\bigskip

\bigskip

\noindent Professor Veronica Guerrieri

\noindent 

\noindent University of Chicago\bigskip

\bigskip

\bigskip

\bigskip

\bigskip

\bigskip

\noindent \textbf{Re: Manuscript \textquotedblleft A Risk-based Theory
  of Exchange Rate Stabilization \textquotedblright }

\bigskip

\noindent Dear Veronica,

\bigskip


\noindent
Please find enclosed an extensive revision of the above-mentioned manuscript. 

We much appreciate the very useful comments by you and the referees.
We have incorporated all of your suggestions and almost all of the
suggestions from the referees in the revised version of the paper.

We would like to thank you for pushing us on various aspects of the
paper. We tightened up loose pieces and added additional analysis. We
believe that the resulting paper is much better than the previous
version. We hope you agree with this assessment.

Here is an overview of the main changes we made following your and the
referees' comments:

\begin{itemize}
\item FIXME
\end{itemize}

\bigskip

More specifically, following the ordering in your letter:

\textit{Referee 1 makes two main recommendations in terms of exposition that are easy
to implement and you should do so ...}

\begin{itemize}
\item FIXME
\end{itemize}

\textit{I recommend you to follow his/her ``less extensive''
  suggestion of using a model of currency market segmentation only in
  the last section of the paper.}

\begin{itemize}
\item FIXME
\end{itemize}

\textit{Finally, Referee 3 is the negative one... In the revision you
  should address this concern and clarify the analysis.}

\begin{itemize}
\item FIXME
\end{itemize}

In addition to your comments above, we incorporated almost all
suggestions from the referees. Our responses to their comments are
summarized at the end of this letter. We found the comments of all
three referees to be very useful and would like to thank them for
their time and for their suggestions.

Thank you very much again for giving us an opportunity to revise the
paper, and we hope that you will find the revision to be responsive to
all of your comments and to almost all of the comments of the
referees. Needless to say, we would be happy to incorporate any
further suggestions you have.


\bigskip

\noindent Tarek, Thomas and Tony

%%%%%%%%%%%%%%%%%%%%%%%%%%%%%%%%%%%%%%%%%%%%%%%%%%%%%%%%%%%%%%%%%%%%%%%%%%%%%%%%
%%% RESPONSE TO REFEREES    
%%%%%%%%%%%%%%%%%%%%%%%%%%%%%%%%%%%%%%%%%%%%%%%%%%%%%%%%%%%%%%%%%%%%%%%%%%%%%%%%

\newpage

\setcounter{page}{1}


%%%%%%%%%%%%%%%%%%%%%%%%%%%%%%%%%%%%%%%%%%%%%%%%%%%%%%%%%%%%%%%%%%%%%%%%%%%%%%%%
%%% REFEREE 1
%%%%%%%%%%%%%%%%%%%%%%%%%%%%%%%%%%%%%%%%%%%%%%%%%%%%%%%%%%%%%%%%%%%%%%%%%%%%%%%%

\section*{Referee 1}
\noindent We would like to thank the referee for their thorough
reading of the previous draft, the careful review, and the excellent
comments they provided. We tried to address all of the comments we
could. As a result, we think our new draft is much cleaner and better
than the previous version. We therefore thank the referee for pushing
us on various aspects of the paper.

\begin{itemize}
\item[1.1] \textit{Change the tone of the introduction. Make it clear
    that you would like to make a conceptual point and chose to do so
    in a standard framework. You inherit all the shortcomings of that
    framework, but none are crucial to your main message that is to
    apply generally.}

\item FIXME

\item[$\rightarrow $] FIXME


\item[1.2] \textit{Please clarify better the relationship between
    firms and exchange rates.}

\item FIXME.

\item[$\rightarrow $] FIXME.


\item[1.3] \textit{Change the last section of the paper to use a model
    of currency market segmentation and compare the results there.}

\item Thank you for this comment. In general, we agree that we should
  clarify the link between models with international currency market
  segmentation from the literature and our setup. Towards this goal, we
  have re-written the model in Section [FIXME] such that each country
  contains a financial intermediary that trades in international bonds
  whereas households only hold domestic bonds.

\item[$\rightarrow $] FIXME.

\end{itemize}


\newpage


%%%%%%%%%%%%%%%%%%%%%%%%%%%%%%%%%%%%%%%%%%%%%%%%%%%%%%%%%%%%%%%%%%%%%%%%%%%%%%%%
%%% REFEREE
%%%%%%%%%%%%%%%%%%%%%%%%%%%%%%%%%%%%%%%%%%%%%%%%%%%%%%%%%%%%%%%%%%%%%%%%%%%%%%%%

\section*{Referee 3}
\noindent We would like to thank the referee for their thorough
reading of the previous draft, the careful review, and the excellent
comments they provided. We tried to address all of the comments we
could. As a result, we think our new draft is much cleaner and better
than the previous version. And we much appreciate your careful review
of our previous draft. We therefore thank the referee for pushing us
on various aspects of the paper.

\begin{itemize}
\item[3.1] \textit{The authors should discuss (a) what happens to
    non-traded good prices under lasseiz faire and tax policy ...}

\item FIXME

\item[$\rightarrow $] Under lasseiz faire, the price of the non-traded
  good in any given country $h$ is:
  \begin{equation*}
    p^{h \ast}_N
    = \frac{(1 - \tau)(\gamma - 1)}{1 + (\gamma - 1) \tau} \bar{y}_N
    - \frac{\gamma}{1 + (\gamma - 1) \tau} y^h_N
    + \log\left( \frac{1 - \tau}{\tau} \right),
  \end{equation*}
  where $\bar{y}_N = \sum \theta^h y^h_N$ is the world average
  production of non-traded goods. Thus, the first term is common
  across all countries. However, the second term is country-specific
  and thus determines the difference in the relative price of
  non-traded goods across countries: If a country $h$ has a relatively
  low production of the non-traded good, $p^h_N$, rises relative to
  $p^f_N$ in all other countries $f$. This relative increase in the
  price of the non-traded good leads to an increase in the real price
  level in $h$, which also implies the payoff of the country $h$
  risk-free bond is high relative to the risk-free bonds in all other
  countries.

  Under the tax policy, the price of the non-traded good in the
  stabilizing country is:
  \begin{equation*}
    p^m_N
    = p^{m \ast}_N
    + \zeta \frac{(\theta^m + (\gamma - 1) \tau)}{\tau (1 + (\gamma - 1) \tau)}
    \left( y^m_N - y^t_N \right).
  \end{equation*}
  Whenever the production of non-traded goods in the target country is
  relatively low, the tax policy leads to an increase in the relative
  price of the non-traded good.


\item[3.2] \textit{... (b) whether deviations from the law of one
    price are also necessary in models with specialized tradables
    hence with terms of trade fluctuations.}

\item[$\rightarrow $] FIXME

\item[3.3] \textit{How can a tax improve the allocation? The claim is
    that each country is a monopolistic supplier of own non-traded
    goods is not convincing.}

\item FIXME

\item[$\rightarrow $] FIXME

\end{itemize}

\newpage

\bibliographystylesec{chicago} \bibliographysec{CUWARS}
